% T I T L E   P A G E
% -------------------
% Last updated Nov 1, 2016, by Stephen Carr, IST-Client Services
% The title page is counted as page `i' but we need to suppress the
% page number.  We also don't want any headers or footers.
\pagestyle{empty}
\pagenumbering{roman}

% The contents of the title page are specified in the "titlepage"
% environment.
\begin{titlepage}
        \begin{center}
        \vspace*{1.0cm}

        \Huge
        {\bf{A modelling investigation into the impacts of the convective parameterization on the tropical circulation} }

        \vspace*{1.0cm}

        \normalsize
        by \\

        \vspace*{1.0cm}

        \Large
        Shawn Corvec\\

        \vspace*{3.0cm}

        \normalsize
        A thesis \\
        presented to the University of Waterloo \\ 
        in fulfillment of the \\
        thesis requirement for the degree of \\
        Master of Mathematics \\
        in \\
        Applied Math \\

        \vspace*{2.0cm}

        Waterloo, Ontario, Canada, 2017 \\

        \vspace*{1.0cm}

        \copyright\ Shawn Corvec 2017 \\
        \end{center}
\end{titlepage}

% The rest of the front pages should contain no headers and be numbered using Roman numerals starting with `ii'
\pagestyle{plain}
\setcounter{page}{2}

\cleardoublepage % Ends the current page and causes all figures and tables that have so far appeared in the input to be printed.
% In a two-sided printing style, it also makes the next page a right-hand (odd-numbered) page, producing a blank page if necessary.
 


% D E C L A R A T I O N   P A G E
% -------------------------------
  % The following is a sample Delaration Page as provided by the GSO
  % December 13th, 2006.  It is designed for an electronic thesis.
  \noindent
I hereby declare that I am the sole author of this thesis. This is a true copy of the thesis, including any required final revisions, as accepted by my examiners.

  \bigskip
  
  \noindent
I understand that my thesis may be made electronically available to the public.

\cleardoublepage

% A B S T R A C T
% ---------------

\begin{center}\textbf{Abstract}\end{center}

Many studies have shown that the tropical circulations (Walker and Hadley circulations) will weaken in a warmer world. This is sometimes attributed to changes in the tropical mean water cycling rate (driven by convective mass flux), which does not increase as fast as boundary layer water vapour in the tropics. However, this theory is only valid for the large scale upward convective mass flux in the tropics, not necessarily to the local circulations, which are not as energetically constrained. Here, we show that there is also a potential regime in which this argument does not hold by simply changing the convective scheme in a climate model. This regime is one in which the tropical mean convective mass flux can actually increase with warming, provided the precipitation efficiency decreases significantly. Our work supports the theory that the uniform tropical mean static stability increase is the physical driver of the weakening of the tropical circulations with climate change, which is mainly driven by the tropical mean SST increase, regardless of the change in strength of convective mass flux. The local changes in tropospheric diabatic heating from heating are shown to influence the magnitude of the weakening of the Walker circulation.

We find that the precipitation efficiency decreases in an increased sea surface temperature AMIP-type experiment using the CAM4 AGCM with an alternate convective scheme using a unique mass flux closure, leading to a plausible scenario where tropical mean convective mass flux may increase, while the large-scale tropical circulations still weaken. While large-scale upward motion and convective mass flux are closely correlated spatially, the nature of this relationship can change in a warmer world if the precipitation efficiency changes. A decrease in precipitation efficiency can allow for increased upward convective mass flux, but the same tropospheric heating rate response, as the increased rate of condensational heating is offset by increased evaporational cooling. A decrease in precipitation efficiency leads to a lower heating rate per unit of upward mass flux due to a compensating increase in evaporation. The large tropical mean evaporation response seen with this scheme allows for stronger tropical mean convective updrafts, especially of the shallow variety, to balance where the evaporational cooling response is maximized. 
\cleardoublepage

% A C K N O W L E D G E M E N T S
% -------------------------------

\begin{center}\textbf{Acknowledgements}\end{center}

I would like to thank Chad Thackeray for helping me setup and configure CESM on SciNet, Ian Folkins for guidance on compiling CAM4 with the IF scheme and for scientific guidance as well. I would also like to thank Christopher Fletcher whose encouragement and mentorship has gone a long way to helping me get this far.
\cleardoublepage

% D E D I C A T I O N
% -------------------

\begin{center}\textbf{Dedication}\end{center}
This thesis is dedicated to my parents, Carol and John, who encouraged me from a young age to follow my passion for meteorology.
\cleardoublepage

% T A B L E   O F   C O N T E N T S
% ---------------------------------
\renewcommand\contentsname{Table of Contents}
\tableofcontents
\cleardoublepage
\phantomsection    % allows hyperref to link to the correct page

% L I S T   O F   T A B L E S
% ---------------------------
\addcontentsline{toc}{chapter}{List of Tables}
\listoftables
\cleardoublepage
\phantomsection		% allows hyperref to link to the correct page

% L I S T   O F   F I G U R E S
% -----------------------------
\addcontentsline{toc}{chapter}{List of Figures}
\listoffigures
\cleardoublepage
\phantomsection		% allows hyperref to link to the correct page

% GLOSSARIES (Lists of definitions, abbreviations, symbols, etc. provided by the glossaries-extra package)
% -----------------------------
%\printglossaries
\cleardoublepage
\phantomsection		% allows hyperref to link to the correct page

% Change page numbering back to Arabic numerals
\pagenumbering{arabic}


% T I T L E   P A G E
% -------------------
% Last updated Nov 1, 2016, by Stephen Carr, IST-Client Services
% The title page is counted as page `i' but we need to suppress the
% page number.  We also don't want any headers or footers.
\pagestyle{empty}
\pagenumbering{roman}

% The contents of the title page are specified in the "titlepage"
% environment.
\begin{titlepage}
        \begin{center}
        \vspace*{1.0cm}

        \Huge
        {\bf{An investigation into the impacts of convective parameterization on the representation of the tropical circulation in a GCM} }

        \vspace*{1.0cm}

        \normalsize
        by \\

        \vspace*{1.0cm}

        \Large
        Shawn Corvec\\

        \vspace*{3.0cm}

        \normalsize
        A thesis \\
        presented to the University of Waterloo \\ 
        in fulfillment of the \\
        thesis requirement for the degree of \\
        Master of Mathematics \\
        in \\
        Applied Math \\

        \vspace*{2.0cm}

        Waterloo, Ontario, Canada, 2017 \\

        \vspace*{1.0cm}

        \copyright\ Shawn Corvec 2017 \\
        \end{center}
\end{titlepage}

% The rest of the front pages should contain no headers and be numbered using Roman numerals starting with `ii'
\pagestyle{plain}
\setcounter{page}{2}

\cleardoublepage % Ends the current page and causes all figures and tables that have so far appeared in the input to be printed.
% In a two-sided printing style, it also makes the next page a right-hand (odd-numbered) page, producing a blank page if necessary.
 


% D E C L A R A T I O N   P A G E
% -------------------------------
  % The following is a sample Delaration Page as provided by the GSO
  % December 13th, 2006.  It is designed for an electronic thesis.
  \noindent
I hereby declare that I am the sole author of this thesis. This is a true copy of the thesis, including any required final revisions, as accepted by my examiners.

  \bigskip
  
  \noindent
I understand that my thesis may be made electronically available to the public.

\cleardoublepage

% A B S T R A C T
% ---------------

\begin{center}\textbf{Abstract}\end{center}

Many studies have shown that the tropical circulations (Walker and Hadley circulations) will weaken in a warmer world. This is attributed to changes in the tropical mean water cycling rate (driven by convective mass flux), which does not increase as fast as boundary layer water vapour. However, the gross hydrological cycle argument is only valid for the overall upward convective mass flux in the tropics, not necessarily the local circulations, which are not as energetically constrained. Here, we show that there is a potential loophole in the hydrological cycle argument and show that by simply changing the convective scheme a climate model can lead to an opposite signed response in tropical mean convective mass flux if the precipitation efficiency decreases significantly. Our work supports the theory that the uniform tropical mean static stability increase is the physical driver of the weakening of the tropical circulations with climate change, which is mainly driven by the tropical mean SST increase, regardless of the change in strength of convective mass flux. The local changes in tropospheric diabatic heating from local precipitation and cloud radiative heating are shown to influence the magnitude of the weakening of the Walker circulation.

We find that the precipitation efficiency decreases in an increased sea surface temperature AMIP-type experiment using the CAM4 AGCM with an alternate convective scheme, leading to a plausible scenario where tropical mean convective mass flux may increase, while the tropical circulations still weaken (measured by large-scale patterns of winds and upward motion). While large-scale upward motion and convective mass flux are closely correlated spatially, the nature of this relationship can change in a warmer world if the precipitation efficiency changes. A decrease in precipitation efficiency can allow for larger upward mass fluxes, but the same tropospheric heating rate response, as the increased rate of condensational heating is offset by increased evaporational cooling, leading to the same net tropospheric heating rate response. In essence, the relationship between the grid-scale upward motion and the sub-grid scale convective mass flux, is mediated by the total diabatic heating rate. A decrease in precipitation efficiency leads to a lower heating rate per unit of upward mass flux due to a compensating increase in evaporation. The decrease in precipitation efficiency is shown to arise from an increase in the ratio of shallow convection to deep convection and the representation of shallow convection in climate models is thought to be important to climate sensitivity
\cleardoublepage

% A C K N O W L E D G E M E N T S
% -------------------------------

\begin{center}\textbf{Acknowledgements}\end{center}

I would like to thank all the little people who made this thesis possible.
\cleardoublepage

% D E D I C A T I O N
% -------------------

\begin{center}\textbf{Dedication}\end{center}

\cleardoublepage

% T A B L E   O F   C O N T E N T S
% ---------------------------------
\renewcommand\contentsname{Table of Contents}
\tableofcontents
\cleardoublepage
\phantomsection    % allows hyperref to link to the correct page

% L I S T   O F   T A B L E S
% ---------------------------
\addcontentsline{toc}{chapter}{List of Tables}
\listoftables
\cleardoublepage
\phantomsection		% allows hyperref to link to the correct page

% L I S T   O F   F I G U R E S
% -----------------------------
\addcontentsline{toc}{chapter}{List of Figures}
\listoffigures
\cleardoublepage
\phantomsection		% allows hyperref to link to the correct page

% GLOSSARIES (Lists of definitions, abbreviations, symbols, etc. provided by the glossaries-extra package)
% -----------------------------
%\printglossaries
\cleardoublepage
\phantomsection		% allows hyperref to link to the correct page

% Change page numbering back to Arabic numerals
\pagenumbering{arabic}


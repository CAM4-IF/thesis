% uWaterloo Thesis Template for LaTeX 
% Last Updated Nov 4, 2016 by Stephen Carr, IST Client Services
% FOR ASSISTANCE, please send mail to rt-IST-CSmathsci@ist.uwaterloo.ca

% Effective October 2006, the University of Waterloo 
% requires electronic thesis submission. See the uWaterloo thesis regulations at
% https://uwaterloo.ca/graduate-studies/thesis.

% DON'T FORGET TO ADD YOUR OWN NAME AND TITLE in the "hyperref" package
% configuration below. THIS INFORMATION GETS EMBEDDED IN THE PDF FINAL PDF DOCUMENT.
% You can view the information if you view Properties of the PDF document.

% Many faculties/departments also require one or more printed
% copies. This template attempts to satisfy both types of output. 
% It is based on the standard "book" document class which provides all necessary 
% sectioning structures and allows multi-part theses.

% DISCLAIMER
% To the best of our knowledge, this template satisfies the current uWaterloo requirements.
% However, it is your responsibility to assure that you have met all 
% requirements of the University and your particular department.
% Many thanks for the feedback from many graduates that assisted the development of this template.

% -----------------------------------------------------------------------

% By default, output is produced that is geared toward generating a PDF 
% version optimized for viewing on an electronic display, including 
% hyperlinks within the PDF.
 
% E.g. to process a thesis called "mythesis.tex" based on this template, run:

% pdflatex mythesis	-- first pass of the pdflatex processor
% bibtex mythesis	-- generates bibliography from .bib data file(s)
% makeindex         -- should be run only if an index is used 
% pdflatex mythesis	-- fixes numbering in cross-references, bibliographic references, glossaries, index, etc.
% pdflatex mythesis	-- fixes numbering in cross-references, bibliographic references, glossaries, index, etc.

% If you use the recommended LaTeX editor, Texmaker, you would open the mythesis.tex
% file, then click the PDFLaTeX button. Then run BibTeX (under the Tools menu).
% Then click the PDFLaTeX button two more times. If you have an index as well,
% you'll need to run MakeIndex from the Tools menu as well, before running pdflatex
% the last two times.

% N.B. The "pdftex" program allows graphics in the following formats to be
% included with the "\includegraphics" command: PNG, PDF, JPEG, TIFF
% Tip 1: Generate your figures and photos in the size you want them to appear
% in your thesis, rather than scaling them with \includegraphics options.
% Tip 2: Any drawings you do should be in scalable vector graphic formats:
% SVG, PNG, WMF, EPS and then converted to PNG or PDF, so they are scalable in
% the final PDF as well.
% Tip 3: Photographs should be cropped and compressed so as not to be too large.

% To create a PDF output that is optimized for double-sided printing: 
%
% 1) comment-out the \documentclass statement in the preamble below, and
% un-comment the second \documentclass line.
%
% 2) change the value assigned below to the boolean variable
% "PrintVersion" from "false" to "true".

% --------------------- Start of Document Preamble -----------------------

% Specify the document class, default style attributes, and page dimensions
% For hyperlinked PDF, suitable for viewing on a computer, use this:
\documentclass[letterpaper,12pt,titlepage,oneside,final]{book}
 
% For PDF, suitable for double-sided printing, change the PrintVersion variable below
% to "true" and use this \documentclass line instead of the one above:
%\documentclass[letterpaper,12pt,titlepage,openright,twoside,final]{book}

% Some LaTeX commands I define for my own nomenclature.
% If you have to, it's better to change nomenclature once here than in a 
% million places throughout your thesis!
\newcommand{\package}[1]{\textbf{#1}} % package names in bold text
\newcommand{\cmmd}[1]{\textbackslash\texttt{#1}} % command name in tt font 
\newcommand{\href}[1]{#1} % does nothing, but defines the command so the
    % print-optimized version will ignore \href tags (redefined by hyperref pkg).
%\newcommand{\texorpdfstring}[2]{#1} % does nothing, but defines the command
% Anything defined here may be redefined by packages added below...

% This package allows if-then-else control structures.
\usepackage{ifthen}

\newboolean{PrintVersion}
\setboolean{PrintVersion}{false} 
% CHANGE THIS VALUE TO "true" as necessary, to improve printed results for hard copies
% by overriding some options of the hyperref package below.

%\usepackage{nomencl} % For a nomenclature (optional; available from ctan.org)
\usepackage{amsmath,amssymb,amstext} % Lots of math symbols and environments
\usepackage[pdftex]{graphicx} % For including graphics N.B. pdftex graphics driver 
\usepackage{float}
% Hyperlinks make it very easy to navigate an electronic document.
% In addition, this is where you should specify the thesis title
% and author as they appear in the properties of the PDF document.
% Use the "hyperref" package 
% N.B. HYPERREF MUST BE THE LAST PACKAGE LOADED; ADD ADDITIONAL PKGS ABOVE
\usepackage{mwe}
\usepackage{subfig}
\usepackage[outdir=./]{epstopdf}
\usepackage{multirow}
\usepackage{natbib}
\usepackage[breaklinks,hidelinks,pdftex,pagebackref=false]{hyperref} % with basic options
		% N.B. pagebackref=true provides links back from the References to the body text. This can cause trouble for printing.
\hypersetup{
    plainpages=false,       % needed if Roman numbers in frontpages
    unicode=false,          % non-Latin characters in Acrobat’s bookmarks
    pdftoolbar=true,        % show Acrobat’s toolbar?
    pdfmenubar=true,        % show Acrobat’s menu?
    pdffitwindow=false,     % window fit to page when opened
    pdfstartview={FitH},    % fits the width of the page to the window
    pdftitle={uWaterloo\ LaTeX\ Thesis\ Template},    % title: CHANGE THIS TEXT!
%    pdfauthor={Author},    % author: CHANGE THIS TEXT! and uncomment this line
%    pdfsubject={Subject},  % subject: CHANGE THIS TEXT! and uncomment this line
%    pdfkeywords={keyword1} {key2} {key3}, % list of keywords, and uncomment this line if desired
    pdfnewwindow=true,      % links in new window
    colorlinks=true,        % false: boxed links; true: colored links
    linkcolor=blue,         % color of internal links
    citecolor=green,        % color of links to bibliography
    filecolor=magenta,      % color of file links
    urlcolor=cyan           % color of external links
}
\ifthenelse{\boolean{PrintVersion}}{   % for improved print quality, change some hyperref options
\hypersetup{	% override some previously defined hyperref options
%    colorlinks,%
    citecolor=black,%
    filecolor=black,%
    linkcolor=black,%
    urlcolor=black}
}{} % end of ifthenelse (no else)

%\usepackage[automake,toc,abbreviations]{glossaries-extra} % Exception to the rule of hyperref being the last add-on package

% Setting up the page margins...
% uWaterloo thesis requirements specify a minimum of 1 inch (72pt) margin at the
% top, bottom, and outside page edges and a 1.125 in. (81pt) gutter
% margin (on binding side). While this is not an issue for electronic
% viewing, a PDF may be printed, and so we have the same page layout for
% both printed and electronic versions, we leave the gutter margin in.
% Set margins to minimum permitted by uWaterloo thesis regulations:
\setlength{\marginparwidth}{0pt} % width of margin notes
% N.B. If margin notes are used, you must adjust \textwidth, \marginparwidth
% and \marginparsep so that the space left between the margin notes and page
% edge is less than 15 mm (0.6 in.)
\setlength{\marginparsep}{0pt} % width of space between body text and margin notes
\setlength{\evensidemargin}{0.125in} % Adds 1/8 in. to binding side of all 
% even-numbered pages when the "twoside" printing option is selected
\setlength{\oddsidemargin}{0.125in} % Adds 1/8 in. to the left of all pages
% when "oneside" printing is selected, and to the left of all odd-numbered
% pages when "twoside" printing is selected
\setlength{\textwidth}{6.375in} % assuming US letter paper (8.5 in. x 11 in.) and 
% side margins as above
\raggedbottom

% The following statement specifies the amount of space between
% paragraphs. Other reasonable specifications are \bigskipamount and \smallskipamount.
\setlength{\parskip}{\medskipamount}

% The following statement controls the line spacing.  The default
% spacing corresponds to good typographic conventions and only slight
% changes (e.g., perhaps "1.2"), if any, should be made.
\renewcommand{\baselinestretch}{1} % this is the default line space setting

% By default, each chapter will start on a recto (right-hand side)
% page.  We also force each section of the front pages to start on 
% a recto page by inserting \cleardoublepage commands.
% In many cases, this will require that the verso page be
% blank and, while it should be counted, a page number should not be
% printed.  The following statements ensure a page number is not
% printed on an otherwise blank verso page.
\let\origdoublepage\cleardoublepage
\newcommand{\clearemptydoublepage}{%
  \clearpage{\pagestyle{empty}\origdoublepage}}
\let\cleardoublepage\clearemptydoublepage
\newcommand{\norm}[1]{\lVert#1\rVert}
% Define Glossary terms (This is properly done here, in the preamble. Could be \input{} from a file...)
% Main glossary entries -- definitions of relevant terminology
%\newglossaryentry{computer}
%{
%name=computer,
%description={A programmable machine that receives input data,
%               stores and manipulates the data, and provides
%               formatted output}
%}

% Nomenclature glossary entries -- New definitions, or unusual terminology
%\newglossary*{nomenclature}{Nomenclature}
%\newglossaryentry{dingledorf}
%{
%type=nomenclature,
%name=dingledorf,
%description={A person of supposed average intelligence who makes incredibly brainless misjudgments}
%}

% List of Abbreviations (abbreviations type is built in to the glossaries-extra package)
%\newabbreviation{aaaaz}{AAAAZ}{American Association of Amature Astronomers and Zoologists}

% List of Symbols
%\newglossary*{symbols}{List of Symbols}
%\newglossaryentry{rvec}
%{
%name={$\mathbf{v}$},
%sort={label},
%type=symbols,
%description={Random vector: a location in n-dimensional Cartesian space, where each dimensional component is %determined by a random process}
%}
 
%\makeglossaries



\newcommand{\overbar}[1]{\mkern 1.5mu\overline{\mkern-1.5mu#1\mkern-1.5mu}\mkern 1.5mu}
%======================================================================
%   L O G I C A L    D O C U M E N T -- the content of your thesis
%======================================================================
\begin{document}

% For a large document, it is a good idea to divide your thesis
% into several files, each one containing one chapter.
% To illustrate this idea, the "front pages" (i.e., title page,
% declaration, borrowers' page, abstract, acknowledgements,
% dedication, table of contents, list of tables, list of figures,
% nomenclature) are contained within the file "uw-ethesis-frontpgs.tex" which is
% included into the document by the following statement.
%----------------------------------------------------------------------
% FRONT MATERIAL
%----------------------------------------------------------------------
% T I T L E   P A G E
% -------------------
% Last updated Nov 1, 2016, by Stephen Carr, IST-Client Services
% The title page is counted as page `i' but we need to suppress the
% page number.  We also don't want any headers or footers.
\pagestyle{empty}
\pagenumbering{roman}

% The contents of the title page are specified in the "titlepage"
% environment.
\begin{titlepage}
        \begin{center}
        \vspace*{1.0cm}

        \Huge
        {\bf{An investigation into the impacts of convective parameterization on the representation of the tropical circulation in a GCM} }

        \vspace*{1.0cm}

        \normalsize
        by \\

        \vspace*{1.0cm}

        \Large
        Shawn Corvec\\

        \vspace*{3.0cm}

        \normalsize
        A thesis \\
        presented to the University of Waterloo \\ 
        in fulfillment of the \\
        thesis requirement for the degree of \\
        Master of Mathematics \\
        in \\
        Applied Math \\

        \vspace*{2.0cm}

        Waterloo, Ontario, Canada, 2017 \\

        \vspace*{1.0cm}

        \copyright\ Shawn Corvec 2017 \\
        \end{center}
\end{titlepage}

% The rest of the front pages should contain no headers and be numbered using Roman numerals starting with `ii'
\pagestyle{plain}
\setcounter{page}{2}

\cleardoublepage % Ends the current page and causes all figures and tables that have so far appeared in the input to be printed.
% In a two-sided printing style, it also makes the next page a right-hand (odd-numbered) page, producing a blank page if necessary.
 


% D E C L A R A T I O N   P A G E
% -------------------------------
  % The following is a sample Delaration Page as provided by the GSO
  % December 13th, 2006.  It is designed for an electronic thesis.
  \noindent
I hereby declare that I am the sole author of this thesis. This is a true copy of the thesis, including any required final revisions, as accepted by my examiners.

  \bigskip
  
  \noindent
I understand that my thesis may be made electronically available to the public.

\cleardoublepage

% A B S T R A C T
% ---------------

\begin{center}\textbf{Abstract}\end{center}

Many studies have shown that the tropical circulations (Walker and Hadley circulations) will weaken in a warmer world. This is attributed to changes in the tropical mean water cycling rate (driven by convective mass flux), which does not increase as fast as boundary layer water vapour. However, the gross hydrological cycle argument is only valid for the overall upward convective mass flux in the tropics, not necessarily the local circulations, which are not as energetically constrained. Here, we show that there is a potential loophole in the hydrological cycle argument and show that by simply changing the convective scheme a climate model can lead to an opposite signed response in tropical mean convective mass flux if the precipitation efficiency decreases significantly. Our work supports the theory that the uniform tropical mean static stability increase is the physical driver of the weakening of the tropical circulations with climate change, which is mainly driven by the tropical mean SST increase, regardless of the change in strength of convective mass flux. The local changes in tropospheric diabatic heating from local precipitation and cloud radiative heating are shown to influence the magnitude of the weakening of the Walker circulation.

We find that the precipitation efficiency decreases in an increased sea surface temperature AMIP-type experiment using the CAM4 AGCM with an alternate convective scheme, leading to a plausible scenario where tropical mean convective mass flux may increase, while the tropical circulations still weaken (measured by large-scale patterns of winds and upward motion). While large-scale upward motion and convective mass flux are closely correlated spatially, the nature of this relationship can change in a warmer world if the precipitation efficiency changes. A decrease in precipitation efficiency can allow for larger upward mass fluxes, but the same tropospheric heating rate response, as the increased rate of condensational heating is offset by increased evaporational cooling, leading to the same net tropospheric heating rate response. In essence, the relationship between the grid-scale upward motion and the sub-grid scale convective mass flux, is mediated by the total diabatic heating rate. A decrease in precipitation efficiency leads to a lower heating rate per unit of upward mass flux due to a compensating increase in evaporation. The decrease in precipitation efficiency is shown to arise from an increase in the ratio of shallow convection to deep convection and the representation of shallow convection in climate models is thought to be important to climate sensitivity
\cleardoublepage

% A C K N O W L E D G E M E N T S
% -------------------------------

\begin{center}\textbf{Acknowledgements}\end{center}

I would like to thank all the little people who made this thesis possible.
\cleardoublepage

% D E D I C A T I O N
% -------------------

\begin{center}\textbf{Dedication}\end{center}

\cleardoublepage

% T A B L E   O F   C O N T E N T S
% ---------------------------------
\renewcommand\contentsname{Table of Contents}
\tableofcontents
\cleardoublepage
\phantomsection    % allows hyperref to link to the correct page

% L I S T   O F   T A B L E S
% ---------------------------
\addcontentsline{toc}{chapter}{List of Tables}
\listoftables
\cleardoublepage
\phantomsection		% allows hyperref to link to the correct page

% L I S T   O F   F I G U R E S
% -----------------------------
\addcontentsline{toc}{chapter}{List of Figures}
\listoffigures
\cleardoublepage
\phantomsection		% allows hyperref to link to the correct page

% GLOSSARIES (Lists of definitions, abbreviations, symbols, etc. provided by the glossaries-extra package)
% -----------------------------
%\printglossaries
\cleardoublepage
\phantomsection		% allows hyperref to link to the correct page

% Change page numbering back to Arabic numerals
\pagenumbering{arabic}

 

%----------------------------------------------------------------------
% MAIN BODY
%----------------------------------------------------------------------
% Because this is a short document, and to reduce the number of files
% needed for this template, the chapters are not separate
% documents as suggested above, but you get the idea. If they were
% separate documents, they would each start with the \chapter command, i.e, 
% do not contain \documentclass or \begin{document} and \end{document} commands.
%======================================================================
\chapter{Introduction}\label{intro}
%======================================================================
\section{Background}\label{background}

The tropics play an integral role in the Earth's climate system; circulations here transport atmospheric momentum, mass, moisture and heat poleward towards the mid-latitudes and drive patterns of rainfall that are vital to billions of people worldwide. These circulations drive and are driven in part by deep moist convection - tropical thunderstorms and downpours - caused by towering plumes of positively buoyant air which derive their buoyancy primarily from the latent heat of condensation. This convection tends to occur over regions where the ocean surface is the warmest (such as in the tropical western Pacific), or over certain land regions during certain times of the year (monsoon season). Convection is what drives a majority of the rainfall in the tropics and patterns of convection are thus vital to the freshwater supply of those living in tropical regions.

As atmospheric concentrations of greenhouse gases increase due to human activities, these patterns of rainfall are projected to change with some areas likely to get wetter and some to get drier. However, there is larger uncertainty in projections of precipitation under climate change are more uncertain than projections of temperature. Even worse, tropical rainfall, which is the majority of the global precipitation, has large uncertainties in highly populated regions such as south Asia. Part of the uncertainty in projections of tropical rainfall under climate change in climate models is related to the fact that moist convection can not be explicitly resolved in today's climate models and needs to be parameterized. This is because convection occurs at very small scales (on the order of 1-10 km) Improving convection schemes in climate models is (or should be) a top priority. This thesis examines a new type of convective scheme under development for use in a version of the NCAR atmospheric climate model (CAM4).
\subsection{Tropical rainfall and the hydrologic cycle}\label{hydro}
Rainfall in the tropics is quite different from what we are used to in mid-latitudes where it is usually associated with large-scale storm systems. While rainfall in the tropics is mainly generated by many localized convective cells \citep{folkins_simple_2014} that can be $\mathcal{O}$(1 km) in size, these cells can have organization in the tropics. There are many different ways the rainfall in the tropics organizes on many different spatial and temporal scales including mesoscale convective systems, Kelvin waves, equatorial Rossby Waves, mixed Rossby-gravity waves, tropical cyclones, monsoons the MJO (Madden-Julian Oscillation) and more. However, it is best to start in the broadest sense with the basic equations for water conservation in the tropical atmosphere. In any given atmospheric column, the vertically integrated flux convergence of water vapor into the column must be equal to the rainfall rate (over sufficiently long timescales):
\begin{align}
P=-\int_{0}^{z}\mathbf{\nabla}\cdot(\rho{q}\mathbf{V})dz+E
\end{align}
(From \citep{holton_introduction_2004}, p.393)
$P$ is the precipitation rate (kg m$^{-2}$), $\rho$ is the density of air (kg m$^{-3})$, $q$ is the water vapor mixing ratio (unitless) and $E$ is the surface evaporation rate (kg m$^{-2}$). The top integration limit, z, could be the entire troposphere, however this is not necessary as most of the moisture in the tropics is in the boundary layer. This is because of the exponential relationship between the amount of water vapor the atmosphere can hold and temperature and also the fact that most of the tropics are ocean and this is where the evaporated ocean water is mixes into ($q$ in the boundary layer is roughly constant with height). So z can be taken to be approximately 2 km (\citep{holton_introduction_2004}, p.393-394). It is also important to relate evaporation over the ocean to known time-mean variables; this is known as the bulk aerodynamic formula for latent heat flux (units of W m$^{-2}$):

\begin{align}
F_{LH}=\rho{L}C_{E}\norm{\textbf{v}}(q_{s}-q_{a})
\end{align}

Where $\rho$ is the time-mean air density of the air at the ocean-atmosphere interface, $L$ is the latent heat of vaporization, $\norm{\textbf{v}}$ is the surface or 10-meter wind speed, $q_{s}$ is the water vapor mixing ratio at the ocean surface which is taken to be the saturation water vapor mixing ratio determined by the sea surface temperature (SST), and $q_{a}$ is the water vapor mixing ratio of the atmosphere just above the ocean surface. $C_{E}$ is an empirically determined non-dimensional constant (sometimes called the drag constant) determined to be $\sim$ 1.1-1.2 x $10^{-3}$ for weak surface winds (which are typical in the tropics) \citep{katsaros_evaporation_2001}. From this equation it is easy to see that the drier and windier the air above the ocean the greater the evaporation rate (latent heat flux). Also, the warmer the SST, the greater the evaporation rate (since $q_{s}$ solely depends on SST). Since SSTs are mainly warmed via incoming shortwave solar radiation in the tropics, any change in incoming solar radiation (clouds, for example) can change the evaporation rate via SST. 

On sufficiently long time scales (certainly those used to study climate) global mean evaporation must balance with precipitation:
\begin{align}
\langle\overbar{P}\rangle-\langle\overbar{E}\rangle=-\Bigg\langle\int_{0}^{z}\mathbf{\nabla}\cdot(\overbar{\rho}{\overbar{q}}\overbar{\mathbf{V}})dz\Bigg\rangle=0
\end{align}
Here, $\langle{.}\rangle$ denotes a global mean (will be used for the zonal mean in other parts of the thesis), and we have added the over-bar to stress that these are time mean quantities (this notation is consistent throughout the thesis). In other words, global time-mean precipitation must balance global time-mean evaporation and the global mean, time-mean moisture flux convergence must also be zero. This is actually a very important constraint or studying climate change as any changes in global mean evaporation will thus change the global mean precipitation rate. In a warming climate, SSTs will warm and thus $q_{s}$ will increase by the Clausius-Clapeyron (C-C) relation:
\begin{align}
\frac{d\ln{e_{s}}}{dT}=\frac{L}{RT^{2}}
\end{align}
$R$ is the gas constant and $e_{s}$ is the saturation vapor pressure, which is directly proportional to the saturation water vapor mixing ration, $q_{s}$. From this relation, it is clear that there is an exponential relationship between saturation vapor pressure (and thus saturation water vapor mixing ratio) and temperature. The rate of increase in $q_s$ as a function of temperature for temperature values seen in the tropics is $\sim$ $7 \%$ K$^{-1}$ based on this relation and is indeed validated by modeling studies \citep{held_robust_2006,vecchi_weakening_2006}. 

While the amount of water vapor in the Earth's atmosphere increases with increasing surface temperature close to the C-C relation, rainfall in climate models does not. This is because global mean evaporation is energetically constrained to increase at a lower rate, and global precipitation must balance evaporation. The evaporation rate is constrained by the amount of downward shortwave (solar) and longwave (infrared) radiation at the surface (which then goes into heating the oceans). In a warming climate, downward infrared flux at the surface increases as the atmosphere gets warmer and emits more infrared radiation upward and downwards which acts to increase the evaporation rate \citep{boer_climate_1993}. The increase in evaporation/precipitation rate with climate change is estimated at 1-3$\%$ K$^{-1}$ \citep{schneider_water_2010,he_anthropogenic_2015,boer_climate_1993}, much less than the $\sim$ $7 \%$ K$^{-1}$ increase for water vapor. The uncertainty in rate of increase of evaporation/precipitation with climate change is likely due to uncertainty in the change in incoming solar radiation due to changes in clouds \citep{allen_constraints_2002}.

\subsection{Moist convection}\label{moist}

Moist convection manifests itself in the tropics and elsewhere as towering and bubbling plumes of cloud called cumulus clouds that can eventually reach the tropopause and spread out into what is referred to as an ``anvil" (composed of ice crystals). Once the cloud forms begins to form an anvil it is referred to as a cumulonimbus cloud and can become electrified due to charge separation and produce lightning. In the tropics, cumulonimbus clouds can easily reach heights of 55000 feet ($\approx$ 17 km) or more \citep{hollars_comparisons_2004}. Tropical cyclones (tropical storms and hurricanes) are driven by organized clusters moist convection that sometimes occurs in the tropics and subtropics. See figure \ref{fig:cumu} for examples of different tropical cumuli clouds in various stages of development.

To better understand the physics of moist convection, a simple model called ``parcel theory" is used in meteorology. This theory makes the simplification that a bubble or ``parcel" of buoyant air (air with a lower density than the environment), is lifted upwards until it condenses and then reaches a level where it is positively buoyant (level of free convection). The integrated potential energy of this parcel above this level is referred to as ``CAPE" (convective available potential energy). One key assumption made in this theory is that the parcel does not dilute due to mixing of environmental air with this ``parcel". This potential energy is then transformed into the kinetic energy of the parcel (assuming the CAPE is positive). Once the parcel reaches its condensation level, the parcel's internal temperature will not fall as fast as the surrounding environment as it ascends, due to the continued latent heat release due to condensation. This assumes that the parcel remains saturated with respect to the environment above this level. Oftentimes, the latent heat of fusion (i.e., from freezing) is ignored when calculating CAPE.
\\
CAPE can be defined as: 
\begin{align}
CAPE=\int_{p_{el}}^{p_{lfc}}(\alpha_{p}-\alpha_{e}){dp}
\end{align}
Where $\alpha_{p}$ is the parcel specific volume (inverse of density) of a parcel above the condensation level, $\alpha_{e}$ is the specific volume of the background atmosphere (as a function of $p$), $p_{lfc}$ is the pressure of the liquid condensation level, and $p_{el}$ is the pressure of the equilibrium level, or level of neutral buoyancy where the parcel reaches zero buoyancy. CAPE is a key parameter for the parameterization of moist convection in a climate or weather model as it is often used for the mass flux closure.

The density (and hence specific volume, $\alpha$) of dry air ($\rho_{dry}$) can be simply calculated using the ideal gas law for use in (1.1):
\begin{align}
\rho_{dry}(p,T)=\frac{p}{R_{d}T}
\end{align}
Where $R_{d}$ is the dry gas constant, $T$ is the temperature, and $P$ is the atmospheric pressure.
\\
\\
However, this neglects the fact that water vapour acts to decrease the air density and thus is sometimes corrected for in calculating CAPE. In place of the specific volume, a quantity referred to as the ``virtual temperature" ($T_{v}$) is introduced (also sometimes referred to as the ``density temperature", $T_{p}$), which is the temperature that a parcel of dry air would have to have in order to have the same density as the parcel of air that has water vapour (at the same pressure). $T_{v}$ is defined as (from: \citep{AMS_virtual_2012}).
\begin{align}
T_{v}=T\frac{(1+\frac{q}{\epsilon})}{(1+q)}
\end{align}
Where q is the water vapour mixing ratio and $\epsilon$ is the ratio of the gas constant for dry air, $R_{d}$ (287 J/K/kg), to the gas constant for water vapour $R_{v}$ (461 J/K/kg) which is $\approx$ 0.623. Using the virtual temperature, the equation for CAPE then becomes \citep{doswell_effect_1994}:
\begin{align}\label{eq:CAPE}
CAPE=g\int_{z_{el}}^{z_{lfc}}\Big(\frac{(T_{v})_{p}-(T_{v})_{e}}{(T_{v})_{e}}\Big){dz}
\end{align}
Note that we have switched the integration to be over geometric height coordinates here from pressure coordinates. 
\\
\\
Even this formulation of CAPE, however, is not perfect as it still neglects dilution of buoyant parcels due to ``entrainment" of drier environmental air and condensate loading (i.e., cloud water and ice that weigh down the parcel) which both act to decrease parcels' buoyancy.
\\
\\
While CAPE is an excellent measure of convective potential, this parameter is rarely directly available as output in climate models and needs to be calculated using a vertical numerical integration at each grid point.
\begin{figure}[H]
\centering
\noindent\includegraphics[width=0.8\linewidth]{../figures/cumuli_photos.jpg}\hfill
\caption{Tropical cumuli in different stages of development. From: \citep{evans_introduction_2011}.}
\label{fig:cumu}
\end{figure}

\subsection{Quasi-equilibrium theory}\label{QET}

In the tropics (30$^\circ$S-30$^\circ$N is often used as a definition of the tropics), CAPE is typically positive, especially over oceans, but why this must be the case is not necessarily obvious. Part of the reason may be the fact that the amount of water vapour the atmosphere can hold increases approximately exponentially with increasing temperature by the Clausius-Clapeyron relation ($\sim{7\%}K^{-1}$), so warmer regions have more water vapour for parcels and thus can release more latent heat and achieve positive buoyancy. \citep{folkins_ian_tropical_2003} found that there is a sharp increase in convective rainfall where sea surface temperatures (SSTs) exceed 26$^{\circ}$C which is typically true in much of the tropics. Because the tropics typically have positive CAPE (i.e., the tropics are convectively unstable), rainfall from moist convection is the predominant source of rainfall in these regions, with ``stratiform" (non-convective) precipitation, typically associated with organized weather systems, much less frequent than in the mid-latitudes.

A model of convection that has been used in convective parameterization schemes in climate and weather models is the model of quasi-equilibrium \citep{arakawa_and_schubert_interaction_1974} (see \cite{emanuel_quasi-equilibrium_2007} for more on this theory). Moist convection, through the latent heat of condensation, acts to warm the environmental atmosphere. In a region where moist convection is frequent, the atmospheric lapse rate tends towards ``moist adiabatic" ($\sim{~6.5}$K/km) which is the lapse rate that a saturated lifted parcel would experience. This acts to reduce the positive buoyancy that rising parcels would experience towards zero. Once this state is achieved, moist convection will no longer occur until the CAPE builds up to become positive again. This is typically achieved by radiative cooling of the free atmosphere in tropics, which occurs on a much longer timescale. This cycle repeats itself and the assumption is that the time-mean CAPE over long time periods does not change, and is consumed rather quickly once it builds up, compared to the time it takes to build up \citep{emanuel_quasi-equilibrium_2007}. This tendency for the tropical temperature profile to relax towards a moist adiabatic profile has been incorporated into the convection schemes of many climate models, however, based on observations, this only appears to hold in regions of heavy convection. In reality, convection in the tropics does not always extend to the tropopause (cumulus congestus) and there is a deviation from the moist adiabat at the melting level \citep{folkins_melting_2013}.



%----------------------------------------------------------------------
\section{The Tropical Circulation}
%----------------------------------------------------------------------
\subsection{Background}\label{wtg}
In the tropics, temperature gradients are relatively smaller than those in the mid-latitudes \citep{sobel_weak_2001}. This is because the Coriolis force weakens as one heads towards the equator, and thus the Rossby number, $R_{o}=\frac{U}{fL}$ (where U and L are charateristic velocity and length scales, and f is the Coriolis parameter $f=2\Omega\sin{\phi}$ (where $\Omega$ is the angular rotation rate of the Earth and $\phi$ is the latitude), is near 1 (\citep{holton_introduction_2004}, p. 388). The Rossby number represents the ratio of the relative importance of inertia to that of the Coriolis force. This number is small in mid-latitudes and thus it is appropriate to make the simplification that the Coriolis force balances the pressure gradient force there. With larger Rossby numbers (closer to 1) seen in the tropics, one can not make this simplification of the momentum equations. To balance the same characteristic velocity, $U$, in the tropics, the geopotential gradients would be an order of magnitude smaller. Since temperature is related to geopotential height, this means that horizontal temperature gradients in the tropics are much smaller than in the mid-latitudes. Therefore, one could neglect the horizontal advection of T as the horizontal gradients of T are small. 

Below is the general (Eulerian) thermodynamic tendency equation in pressure coordinates (adapted from p.387 \citep{holton_introduction_2004}):

\begin{align}
\Big(\frac{\partial}{\partial{t}}+\mathbf{V}\boldsymbol{\cdot} \mathbf{\nabla}\Big)T - \sigma\omega =Q_{tot}
\end{align}
\\
Where $\mathbf{v}$ is the total vector wind, $T$ is the temperature, $\sigma$ is the static stability ($\sigma=\frac{RT}{c_{p}p} - \frac{\partial{T}}{\partial{p}} = \frac{T}{\theta}\frac{\partial{\theta}}{\partial{p}}$), $\omega$ is the upward motion motion, $\frac{dp}{dt}$, and $Q_{tot}$ is the total diabatic heating rate. 
\\
\\
$\omega$ is a Lagrangian quantity that is defined as (\cite{holton_introduction_2004} p. 75):
\begin{align}\label{omega}
\omega=\frac{dp}{dt}
\end{align}
Where a negative (positive) value indicates upward (downward) motion (because pressure decreases with height). Note that the $\frac{d}{dt}$ here is the total, or material derivative (in pressure co-ordinates): $\frac{\partial}{\partial{t}} + \mathbf{V}\cdot\mathbf{\nabla}p + w\frac{\partial{q}}{\partial{z}}$, where $\mathbf{\nabla}$ is the horizontal del operator and $w$ is the vertical wind.
\\
\\
Neglecting horizontal T advection in the tropics for the reasons discussed above gives:

\begin{align}
-\sigma\omega \approx{Q_{tot}}
\end{align}
Where
\begin{equation}\label{eq:sp}
  \sigma\omega\equiv-\omega\frac{T}{\theta}\frac{\partial{\theta}}{\partial{p}}
\end{equation}
\\
We have introduced the potential temperature, $\theta$, in place of $T$ which has similarities to density and is defined as the temperature that a parcel of air would have if moved adiabatically to a reference pressure, $p_{o}$:
\begin{align}
\theta=T\Big(\frac{p_{o}}{p}\Big)^{(\frac{R_{d}}{c_{p}})}
\end{align}
\\
(1.6) implies that the total diabatic heating at any point in the tropics is balanced by vertical temperature advection (which is upward motion, $\omega$, multiplied by the static stability, $\sigma$).
This simple relation is vital to understanding tropical dynamics and the tropical circulations which will be discussed in the next sections.

\subsection{Walker circulation}\label{WC}

The Walker circulation is named after Sir Gilbert Walker who while working in India noticed that occasionally the summer monsoon rains would not materialize. He discovered that this seemed to be related to sea level pressure (SLP) gradient changes between the western and eastern equatorial Pacific. The oscillation of this pressure gradient is referred to as the ``southern oscillation" and is typically calculated as a difference in SLP between Darwin, Australia and Tahiti, in the central Pacific (\citep{holton_introduction_2004} p.382-383). Typically, the SLP over the Maritime Continent and far western tropical Pacific is lower than the SLP in the central/eastern Pacific, but this pressure difference can shrink or even reverse during El Nino events. This pressure gradient causes winds to blow from east to west in the tropical Pacific (trade winds), with rising motion and low-level convergence over the Maritime Continent (Indonesia and environs) and sinking motion and low-level divergence over the eastern Pacific \citep{bjerknes_atmospheric_1969}. The rising motion and upper-level divergence over the Maritime Continent is caused by the relatively warmer waters (sometimes referred to as the western Pacific warm pool) in that region which creates an environment of positive CAPE, and thus convection is favoured there. This region of zonally enhanced time-mean convection generates significant condensational (and freezing) heating in the mid and upper troposphere which acts to rise geopotential heights and create a horizontal pressure gradient out of the region and thus creates a region of upper-level divergent winds flowing outward horizontally from the region. By mass continuity, air from below moves upward and a compensating region of descending air and upper-level convergence forms in the eastern Pacific where covnection is much less frequent. Some describe the Walker Circulation as a standing Kelvin wave being forced by diabatic heating \citep{stechmann_walker_2014}. 

One way to measure the strength of the Walker circulation is to measure upper-tropospheric horizontal divergence (which gives information about the vertical motion by continuity). However, this is typically a very noisy field so instead a scalar potential field of the divergent wind is typically used. This field is obtained using the fundamental theorem of vector calculus (also known as a Helmholtz decomposition) which states that a vector field can be decomposed into divergenceless (rotational) and irrotational (divergent) components. Stated mathematically, the total vector wind, $\mathbf{V}$, can be written as:
\begin{align}
\mathbf{V}=\mathbf{V_{rot}}+\mathbf{V_{irr}}
\\
\mathbf{\nabla} \times \mathbf{V_{irr}} = 0
\\
\mathbf{\nabla} \boldsymbol{\cdot}\mathbf{V_{rot}} = 0
\end{align}
From these 2 components, a stream function and a velocity potential can be constructed.
A velocity potential, $\chi$ can be constructed from the divergent (irrotational) component of the wind so that:
\begin{align}\label{eq:chi}
\mathbf{\nabla}\chi=\mathbf{V_{irr}}
\end{align}
Which satisfies (1.9) since the curl of the gradient of a scalar field is 0. 
\\
\\
This scalar field evaluated in the upper troposphere, such as at the 200 hPa level, is often used for measuring the strength of the upper-level divergence associated with the ascending region of the Walker circulation \citep{tanaka_trend_2004}. Since the Walker circulation is a zonally asymmetric circulation, the zonal mean needs to be removed to ascertain the contribution Walker circulation contribution to the divergence. This will be referred to as $\chi^{*}$ in this thesis, where $\cdot^{*}$ indicates a deviation from the zonal mean, and this notation will be consistent through this thesis. 

The zonal asymmetry in diabatic heating in the tropics is what causes the zonal asymmetry in divergence and hence the zonal (Walker circulation). This asymmetry is due to the fact that SSTs are warmer in the western Pacific than in the eastern Pacific. The primary balance for the Walker circulation is between diabatic heating and adiabatic cooling in the ascending region (eq. 1.6) and between radiational cooling and adiabatic heating in the descending region (also, eq. 1.6). Thus, any changes to the rate of diabatic heating, $Q_{tot}$ or static stability, $\sigma$ can effect the strength of the circulation, $\omega$.
\subsection{Hadley circulation}\label{HC}

The Hadley circulation is a zonally-symmetric meridional (north-south) circulation that is characterized by rising air near the equator that reaches the upper troposphere. This circulation is driven by general poleward geopotential height gradient through the troposphere driven by the difference in solar heating between the tropics and mid-latitudes. As air from the tropics heads poleward, it gains in velocity to conserve angular momentum and is deflected eastward due to the Coriolis force. While the Walker circulation can be modeled through a simple energy balance equation, because the Hadley circulation extends into the sub-tropics and beyond, the weak temperature approximation is no longer valid and the Coriolis force no longer small (i.e., the Rossby number, $R_{o}$, is smaller). This complication means that other factors not directly linked to the convection scheme such as the equator-to-pole temperature gradient are important for the Hadley circulation strength and thus most of the analysis of the tropical circulation in this thesis will be for the Walker circulation.
\\
\\
While the Walker circulation can be understood as a balance between diabatic heating (cooling) and adiabatic (heating), the Hadley circulation is more complicated. For the Hadley circulation the balance is now between the Coriolis force plus the divergence of eddy momentum fluxes, as well as between diabatic heating and adiabatic cooling plus the divergence of eddy heat fluxes. Next, I will briefly describe how these relations are obtained, and what factor influence the strength of the Hadley circulation. 
\\
\\
Following \citep{holton_introduction_2004}, p.318, we start with the zonal mean zonal momentum and zonal mean (zonal means in this section are indicated by $\langle \cdot \rangle$ notation) temperature tendency equations assuming quasi-geostrophic motion on a $\beta$ plane (Coriolis parameter is assumed to vary linearly with latitude):

\begin{align}
\frac{\partial{\langle{u}\rangle}}{\partial{t}} - f_{o}\langle{v}\rangle = -\frac{\partial\Big(\langle{u^{*}v^{*}}\rangle\Big)}{\partial{y}} + \langle\text{frictional drag}\rangle
\end{align}
\begin{align}
\frac{\partial{\langle{T}\rangle}}{\partial{t}} + \frac{N^{2}H}{R}\langle{\omega}\rangle = -\frac{\partial\Big(\langle{u^{*}T^{*}}\rangle\Big)}{\partial{y}} + \langle{Q_{tot}}\rangle
\end{align}
$N$ is the buoyancy or Brunt-Vaisala frequency which can be defined as: 
\begin{align}
N\equiv\sqrt{\frac{g}{\theta}\frac{d{\theta}}{dz}}
\end{align}
And is similar to static stability, $\sigma$, in that it is a measure of the stratification.
\\
\\
If we assume that the zonal mean meridional wind, $\langle{v}\rangle$, and the zonal mean temperature, $\langle{T}\rangle$ are in equilibrium ($\frac{\partial}{\partial{t}}=0$) and neglect the effect of frictional drag, (1.13) and (1.14) simplify to:
\begin{align}
-f_{o}\langle{v}\rangle \approx -\frac{\partial\Big(\langle{u^{*}v^{*}}\rangle\Big)}{\partial{y}}
\end{align}
\begin{align}
\frac{N^{2}H}{R}\langle{\omega}\rangle \approx -\frac{\partial\Big(\langle{u^{*}T^{*}}\rangle\Big)}{\partial{y}} + \langle{Q_{tot}}\rangle
\end{align}
\\
\\
So, in equilibrium, the zonal mean meridional flow is proportional to the divergence of the zonal mean eddy momentum flux ($\langle{u^{*}v^{*}}\rangle$) and zonal mean upward motion ($\langle{\omega}\rangle$) is proportional to the stratification times the zonal mean diabatic heating and zonal mean eddy heat flux convergence ($\langle{u^{*}T^{*}}\rangle$) terms. Note that if the eddy heat flux convergence is considered to be negligible we return the balance between diabatic heating and adiabatic cooling (1.6) seen for the Walker circulation. 
\\
\\
Exploiting the fact that the zonal mean meridional circulation is nondivergent (\citep{holton_introduction_2004}, p.319), one can construct a zonal mean stream function, $\Psi$, that represents the vertical and meridional flow. This stream function can be calculated solely from the zonal mean meridional wind and is referred to as the meridional mass streamfunction:
\begin{align}
\Psi=\frac{2\pi\cos{\phi}}{g}\int_{p_{top}}^{p} \langle{v}\rangle{dp}
\end{align}
Where
\begin{equation}\label{eq:test}
\langle{v}\rangle \propto -\frac{\partial{\Psi}}{\partial{z}} \text{ and } \langle{\omega}\rangle \propto\frac{\partial{\Psi}}{\partial{y}}
\end{equation}
\\
\\
Combining (1.13) and (1.14) one can obtain a relation between zonal mean stream function and forcing terms. This equation is analgous to the omega equation used in quasi-geostrophic theory in mid-latitude Meteorology. From \citep{holton_introduction_2004}, p.320:
\begin{equation}
\begin{split}
\Psi \propto -\frac{\partial}{\partial{y}}(\langle\text{diabatic heating}\rangle) + \frac{\partial^{2}}{\partial{y^{2}}}(\langle\text{eddy heat flux}\rangle) \\ + \frac{\partial^{2}}{\partial{y}\partial{z}}(\langle\text{eddy momemtnum flux}\rangle) + \frac{\partial}{\partial{z}}(\langle\text{zonal frictional drag}\rangle) 
\end{split}
\end{equation}
The diabatic heating profile is positive in the tropics and decreases (increases) as one heads north (south), therefore the first term in (1.20) causes $\Psi$ to be positive in the Northern Hemisphere and negative in the Southern Hemisphere (figure \ref{fig:HC}).

\begin{figure}[H]
\centering
\noindent\includegraphics[width=1\linewidth]{../figures/fig1.png}\hfill
\caption{\textbf{a)}: Annual mean latitude vs. height plot of $Q_{tot}$ from the CAM4 AGCM. Note the maximum of diabatic heating near the equator. The region where the diabatic heating declines the fastest with increasing latitude is around where the maximum of streamfunction for each of the 2 cells is located in \textbf{b)} (around 15S/N). Note the vertical scales are not identical.}
\label{fig:HC}
\end{figure}

The seasonal maximum of the diabatic heating term is reached in boreal winter (DJF in NH and JJA in SH), and this is when the Hadley circulation is at its strongest. The contributions from the other terms are much more complicated to understand, however, and unfortunately this makes developing simple tests to see how the convective scheme (which primarily effects the distribution of tropical diabatic heating) effects the strength of the Hadley circulation. The eddy heat and momentum fluxes are mainly driven by higher latitude dynamics and it is unclear how a change in tropical static stability or diabatic heating would effect these dynamics. Although, the equator-to-pole temperature gradient is likely important for the eddy heat transport term as this is the only way to reduce this gradient (\citep{holton_introduction_2004}, p.321). To add even more difficulty, \citep{kim_hadley_2001} points out that the effects of eddies can influence tropical diabatic heating and vice-versa.
\subsection{Response to climate change}\label{response}
Global mean precipitation is expected to increase with climate change as the downwelling longwave radiation increases as the temperature of the atmosphere increases \citep{boer_climate_1993}, thus one might expect the tropical circulations to increase in strength as well due to the increased condensational heating. However, this has not been found to be the case with global climate models. The first study to examine the mechanisms behind the weakening of the tropical circulation with climate change was \citep{knutson_time-mean_1995} who showed that the increased tropospheric stratification (static stability) more than offsets the increased diabatic heating means the ascending regions (regions with negative $\omega$) of the tropical circulations must weaken (from 1.13). The tropical mean static stability increases due to the maximum of warming in the tropical upper troposphere because of the shift in the moist adiabatic lapse rate (the temperature profile in the tropics is nearly moist adiabatic).

More recent studies have examined other physical mechanisms for the weakening of the tropical circulation due to global warming. One popular mechanism is the weakening of upward convective mass flux due to hydrological cycle constraints \citep{held_robust_2006,vecchi_global_2007,chadwick_spatial_2012}. This is found to mainly manifest itself in the weakening of the Walker circulation \citep{held_robust_2006,vecchi_global_2007} and \citep{he_anthropogenic_2015} found that the main driver is the mean increase in SST in the CMIP5 models \citep{taylor_overview_2011}. The mechanism is as follows: while the boundary layer specific humidity must increase at a rate dictated by the Clausius-Clapeyron relation ($\approx 7\%K^{-1}$), the global mean evporation rate is contrained to increase at a much slower rate (see section 1.1.1). Assuming that most of the moisture in convective plumes is precipitated out (i.e., parcels from the boundary layer reach all the way to the level of neutral buoyancy with the only loss of water content from precipitation), this implies the upward convective mass flux must slow down. In other words, the updrafts are not as fast, but carry more water mass and thus the precipitation rate can be the same as in a cooler climate with faster updrafts. Using this simple model, a relation can be developed to estimate the tropical mean convective mass flux, 
$M_{c}$ \citep{held_robust_2006}:

\begin{equation}\label{eq:HS}
\Bigg\langle\frac{\delta{M^{'}}}{M^{'}}\Bigg\rangle=\Bigg\langle\frac{\delta{P}}{P}\Bigg\rangle-\Bigg\langle\frac{\delta{q_{bl}}}{q_{bl}}\Bigg\rangle
\end{equation}

Where $M^{'}$ is the inferred convective mass flux, $P$ is the precipitation rate and $q_{bl}$ is the boundary layer mixing ratio. Here, angle brackets represent the tropical mean. The term on the right, representing the change in the amount of moisture the atmosphere can hold, is directly related to the C-C relation and can be estimated as 0.07 per K or warming, while the precipitation term has a bit more uncertainty of $\sim$ 0.01-0.03 per K of warming. This implies that $\langle\frac{\delta{M}}{M}\rangle$ will be negative in a warmer world; in other words, the convective mass flux in the tropics will decrease. 

However, this assumes that a majority of the convective mass flux in the tropics are contained in so-called ``hot-towers" \citep{riehl_and_malkus_heat_1958} which transport parcels rapidly from the boundary layer to the tropopause with no change in moist static energy. However, recent modeling studies using cloud-resolving models show that there is unlikely to be many parcels that make it to the tropopause undiluted \citep{romps_undiluted_2010} as a significant amount of detrainment occurs above the boundary layer \citep{romps_direct_2010}. 

To create even more complications, shallow convection and stratiform rainfall in the tropics accounts for a significant proportion of total rainfall in the tropics \citep{schumacher_stratiform_2003}. In fact, shallow convection has been shown to have its own closed circulation in the tropics \citep{folkins_ian_low-level_2008} which could open a pathway for water vapor to be recharged into the boundary layer instead of all being rained out by deep convection. If the efficiency of rainfall generation per amount of convective mass flux changes with climate change, it could, if the change is large enough, mean that convective mass flux can increase with climate change. A more general potential flaw in this line of reasoning to explain the weakening of the Hadley and Walker circulations is that while the overall upward motion may decrease in the tropical mean, it does not necessarily mean that the local circulations will necessarily weaken \citep{merlis_changes_2011} because local precipitation changes can influence the circulation strength. Local precipitation can increase slower or faster with climate change than the tropical or global mean constraint which arises from the constraint in the increase in the global mean evaporation rate with climate change. 

In some studies, the pressure vertical velocity, $\omega$, is used as a proxy for the convective mass flux (such as in \cite{vecchi_global_2007,schneider_water_2010}). 
%%%%%%%%%%%%%%%%%%%%%%%%%%%%%%%%%%%%%%%%%%%%%
%%%%%%%%%%%%%Fix this section%%%%%%%%%%%%%%
It is not clear physically why upward $\omega$, which is influenced by the large scale circulation, should be thought of as being identical to the subgrid scale convective mass flux which is a parameterized quantity. In fact, \citep{yano_deep-convective_2009} discusses the fact that there is some confusion as to what the convective mass flux actually is. It was first realized by \citep{riehl_and_malkus_heat_1958} that the upward motions of the tropical circulations could not be responsible for the vertical moist static energy structure observed in the tropics with a minimum in the mid-troposphere. Thus, they hypothesized that ``hot towers" plumes of relatively undiluted air could transport tracers from the boundary layer to the tropopause quite efficiently. These convective plumes occur on the sub-gridscale and must be parameterize and should not be thought as being included in $\omega$ which is calculated from the heating tendencies generated from the convective mass flux. In order to maintain hydrostatic balance in the grid box the positive acceleration (buoyancy) is offset by a compensating downward acceleration in the rest of the grid box to ensure the mean vertical acceleration in the grid box is 0, maintaining hydrostatic balance (Ian Folkins, personal communication).

While the convective mass flux can effect the vertical diabatic heating (temperature tendency) structure in an atmospheric column, which can in turn effect the large-scale $\omega$ (see \ref{eq:sp}), it should not be assumed that the $M_{c}$=$\omega$ in the tropics. The spatial correlation between the two is quite high, however (as we have shown in figure \ref{fig:mcomeg}), with values of upward motion and convective mass flux higher in regions of high precipitation than low precipitation in the tropics. Our simulations with the IF scheme in the CAM4 model show that in fact the tropical mean convective mass flux can increase, while the tropical mean ascent and descent can weaken.

Another mechanism used to explain the weakening of the Walker circulation from climate change (similar to the static stability argument) is one using the concept of ``gross moist stability", which is outlined in \citep{wills_local_2017}. This theory starts with the assumption that the Walker circulation strength is simply the zonally-anomalous total energy input into the ascending region ($Q_{tot}^{*})$ divided by the gross moist stability. The gross moist stability of a column can be defined as the pressure-weighted vertical integral of the vertical advection of MSE (moist static energy, hereafter referred to as $h$) in pressure coordinates:
\begin{equation}\label{eq:GMS}
GMS(x,y) = \int_{p_{LCL}}^{p_{trop}}\bigg(\frac{\partial({h(x,y,p)})}{\partial{p}}\omega(x,y,p)\bigg){dp}
\end{equation}
\begin{equation}\label{eq:MSE}
h=C_{p}T + gz + L_{v}q
\end{equation}
The integral is taken from the liquid condensation level (or level of free convection), to the tropopause. Outside of these regions, vertical motions are small \citep{wills_local_2017}. This represents the effective stability that is felt by the convection and is a measure of the efficiency of convection to transport $h$ vertically. $C_{p}$ is the specific heat at constant pressure, $z$ is the geometric height above mean sea level or some other reference height, $L_{v}$ is the latent heat of vaporiztion and $q$ is the water vapour mixing ratio. The authors go on to show that GMS will increase in a warmer world if one makes the approximation that the vertical derivative of $h$ becomes simply: $\Delta{h} = h_{trop} - h_{LCL}$. They go on to show the increase in the term $gz$ dominates in global warming, and thus causes $\Delta{h}$ and thus GMS to increase with global warming and the Walker circulation becomes more ``efficient" at transporting $h$. In other words, the increase in tropopause height and thus the depth of the convection causes the GMS to increase and thus the circulation weakens because it can transport more energy per unit of mass flux. One potentially flaw, that the authors do point out, is that this says nothing about the changes in zonally anomalous energy input (total diabatic heating) into the ascending region from processes such as cloud radiative heating which may change in warmer climate.

While, there is nearly unanimous consensus among the CMIP5 models of a weakening of convective mass flux \citep{chadwick_spatial_2012} and the tropical circulations, with the weakening of the Walker circulation being the most robust \citep{he_anthropogenic_2015}, observational studies have shown that the Walker circulation may in fact be strengthening \citep{lheureux_recent_2013,sandeep_pacific_2014}, while others have shown weakening \citep{vecchi_weakening_2006,power_what_2011}. Additionally, modeling studies have shown little relationship between global mean convective mass flux and the strength of the Walker circulation \citep{sandeep_pacific_2014}. Throughout many of the reanalysis and modeling studies there is often mention that the convective parameterization could be a key factor and influencing the simulated trend in Walker circulation strength. This thesis attempts to determine the sensitivity of the simulated change in Walker circulation strength to the convective scheme.  

The problem of the change in Hadley cell (HC) strength and extent due to climate change appears to be a more complex problem than that of the Walker circulation. The HC is not only effected by the zonal-mean diabatic heating contrast (see eq. 1.26), but also by eddy momentum and heat fluxes, which originate from extratropical regions \citep{walker_eddy_2006,kim_hadley_2001}. The HC is expected to weaken and expand poleward with climate change, with the trend in strength being more uncertain than the expansion \citep{he_anthropogenic_2015,vecchi_global_2007,bony_robust_2013,gastineau_hadley_2009,ma_mechanisms_2011,lu_expansion_2007}. \citep{seo_mechanism_2014} showed that tropics to mid-latitudes temperature gradient explained at least some of the inter-model spread in the change of HC strength 
in the CMIP5 coupled models. However, recent observational studies indicate that while the HC appears to have expanded at least a few degrees north since the beginning of the satellite era (1979) \citep{johanson_hadley_2009,seidel_recent_2007,hu_observed_2007}, it also appears to show signs of a slight strengthening as well \citep{mitas_has_2005,hu_observed_2007,stachnik_comparison_2011}, although there is a very large spread in uncertainty in the HC intensity among the various reanalysis datasets \citep{stachnik_comparison_2011}. It has been shown that ENSO cycles (El Nino and La Nina events) can effect the strength of the winter HC \citep{oort_observed_1996,quan_change_2004}, which gives support to the theories that show HC strength is mainly a function of diabatic heating such as the Held-Hou model \citep{held_nonlinear_1980} which assumes a Rossby number, $R_{o}$, near 1. This is because El Nino events simply re-arrange the distribution of diabatic heating in the tropics by shifting SSTs which also shift patterns of tropical convection and thus condensational heating. \citep{caballero_role_2007} shows that the Rossby number for the winter cell is in fact $\approx$ 0.5, so the winter cell is likely in a regime where both diabatic heating and eddy fluxes from the mid-latitudes are important. Because of these complications, this thesis will focus mainly on the Walker circulation and will only briefly discuss the HC.


%======================================================================
\chapter{Climate model simulation background}
%======================================================================
\section{CAM4 AGCM}\label{cam4}
To perform the simulations for this thesis, the CAM4 (Community Atmosphere Model) general circulation model from NCAR (\url{http://www.cesm.ucar.edu/models/ccsm4.0/cam/}) was used \citep{neale_mean_2013}. A technical description of the model can be found here: (\url{http://www.cesm.ucar.edu/models/ccsm4.0/cam/docs/description/cam4_desc.pdf}). The CAM4 is a global, hydrostatic, full atmosphere model using an Eulerian dynamical core and is typically used for climate simulations, but has no ocean model component for atmosphere-ocean coupling. The CAM4 is part of a larger model package, which includes ocean and ice sheet models, called the CESM (Community Earth System Model, \url{http://www.cesm.ucar.edu/models/}). For this thesis we will perform atmosphere-only experiments using version 4 of the CAM (CAM4) from version 1.0.2 of the CESM (release summary here: \url{http://www.cesm.ucar.edu/models/cesm1.0/tags/#CESM1_0_2}). The CESM (and its predecessor, CCSM) is included in the CMIP3/5 suite of models used for the IPCC \citep{taylor_overview_2011}.

\subsection{Zhang-Macfarlane Scheme}\label{ZM}
To simulate convection in a global climate model with a horizontal resolution large than $\mathcal{O}$( 10 km), a parameterization scheme is needed as the convective cells would be smaller than the grid spacing. This scheme uses ``mass flux closure" to relate large scale environmental variables to the sub-grid scale convective plume updrafts and vice-versa. The large scale variables in each grid box (convective mass flux, relative humidity, temperature, etc.) take on the grid-box average of the sub-grid variables. In other words, the scheme uses mass flux closure to determine a unique grid-box averaged convective mass flux given the grid-box environmental variables. The scheme used in the CAM4 (ZM or ``Zhang-Macfarlane" scheme) \citep{zhang_sensitivity_1995} uses CAPE (see \ref{eq:CAPE}) as the (sole) environmental variable for mass flux closure. The CAM4 actually uses 2 convective schemes; the main one for deep convection and the other for shallow convection. This shallow scheme, the ``Hack" scheme \citep{hack_parameterization_1994} handles shallower convective plumes that do not span the whole depth of the troposphere. In this section we will just focus on the ZM scheme, however.

The ZM scheme uses an ensemble ``plume" approach where a spectrum of updrafts are created in each grid box with the same cloud base updraft mass flux ($M_{cb}$) (p. 89, \citep{neale_description_2010}). Each plume has a different total fractional entrainment rate (i.e., the rate of mixing of mass of drier environmental air into the plume), $\lambda$, and thus above the cloud base the plumes will have a variety of different updraft mass fluxes ($\lambda$ appears to represent the total amount of entrainment through the entire plume). This means that this is a dilute plume parameterization, with dilution with the environment taken into account. Detrainment of the plume into the environment is limited to the very top of the convective plumes and this height is defined as $z_{D}$. The updraft mass flux, $M_{u}$, for the ensemble mean is defined as a function of $z$ (p. 90, \citep{neale_description_2010,zhang_sensitivity_1995}):
\begin{equation}\label{eq:updraft}
M_{u}(z)=M_{cb}\Bigg(\frac{e^{\lambda_{D}}(z_{D}-z)-1}{\lambda_{0}(z_{D}-z)}\Bigg)
\end{equation}
$\lambda_{D}$ is the specific entrainment rate of of that plume at a given height, $z$, which is a function of the total factional entrainment rate $\lambda$ assigned to the plume, with smaller values of $\lambda$ reaching higher $z_{d}$ because less buoyancy is lost in the plume. $\lambda_{0}$ is the detrainment rate for the shallowest convective plumes and is the largest entrainment rate in the ensemble. All that is needed is to find $M_{cb}$ and assign a $\lambda$ to each of the plumes in the ensemble. To calculate the cloud base updraft mass flux in the ZM scheme, a simple closure relation is used (from p. 93, \citep{neale_description_2010}):
\begin{equation}\label{eq:closure}
\frac{\partial{(\text{CAPE})}}{\partial{t}}=-M_{cb}F
\end{equation}
Where
\begin{equation}\label{eq:closure1}
M_{cb}=\frac{\text{CAPE}}{\tau{F}}
\end{equation}
Where $F$ is the rate of consumption of CAPE per unit of $M_{cb}$ and $\tau$ is the timescale for CAPE consumption. Using these 2 relations we can define the CAPE consumption rate as:
\begin{equation}\label{eq:closure2}
\frac{\partial{(\text{CAPE})}}{\partial{t}}=\frac{\text{CAPE}}{\tau}
\end{equation}
These relations provides a simple relationship between CAPE and the starting (cloud base) mass flux in the CAM4 model. Any changes in CAPE are thus due to the convective mass flux are due to heating/moistening processes from the release of CAPE as the convective plume evolves, and thus the relation is closed. In this way, CAPE determines mass flux and mass flux then changes CAPE. Moist convection will only occur when CAPE is positive in this scheme.

The deep convective precipitation production, $P_{prod}$, can be determined from the updraft mass flux as \citep{zhang_sensitivity_1995}:
\begin{equation}\label{eq:pprod}
P_{prod}(z)=C_{0}M_{u}(z)l(z)
\end{equation}
Where $C_{0}=2\cdot{10^{-3}} m^{-1}$ is a liquid to rain conversion constant, and $l$ is the liquid water content. Thus, the deep convective precipitation production in this scheme is linearly proportional to the updraft mass flux. An important approximation made in this scheme is that $P_{prod}$ below the freezing level is 0, and thus the plumes must extend above this level to produce precipitation \citep{zhang_sensitivity_1995}.

The downdraft parameterization in this scheme is the similar to that as for updrafts to originate only where $P_{prod} > 0$, which would be above the freezing level only. The ensemble mean downdraft mass flux, $M_{d}$ as a function of z is given as \citep{zhang_sensitivity_1995}:
\begin{equation}\label{eq:down}
M_{d}(z)=\alpha{M_{cb}}\Bigg(\frac{e^{\lambda_{D}}(z_{D}-z)-1}{\lambda_{0}(z_{D}-z)}\Bigg)
\end{equation}
Where the variables here are the same as before, with $\alpha$ being a proportionality constant to ensure that the net cloud base mass flux of the ensemble is positive and is defined as:
\begin{equation}
\alpha=0.2\Bigg(\frac{[P_{prod}]}{[P_{prod}]-[E]}\Bigg)
\end{equation}
Where $[P_{prod}]$ is the vertically integrated precipitation production in the cloud and $[E]$ is the vertically integrated evaporation. The largest value $\alpha$ can have is 0.2, and the lowest is 0 if there is not precipitation. This relation ensures that the downdraft mass flux is no more than 0.65 of the updraft mass flux \citep{zhang_sensitivity_1995}. The convective precipitation evaporation rate locally in the CAM4 is simply a function of relative humidity and is defined as (p.102 \cite{neale_description_2010}):
\begin{equation}\label{eq:evap}
E=K_{E}(1-RH)(P_{flux})^{0.5}
\end{equation}
$K_{E}$ is a constant (taken to be 0.2 $\cdot$ 10$^{-5}$ (kg m$^{-2}$ s$^{-1}$)$^{-0.5}$s$^{-1}$), $RH$ is the relative humidity, and $P_{flux}$ is the local convective rainfall flux which is a total of rainfall received from levels above. As the rainfall rate increases, the evaporation rate will go up as the square root of the rainfall rate, and the evaporation rate will also increase as relative humidity decreases. Thus, downdrafts are stronger for regions which have higher rainfall rates, for the same relative humidity. Note that the convective scheme parameters $\tau$, $K_{E}$, $\alpha$ and $C_{0}$ are somewhat ``tunable" and some studies (such as \cite{yang_uncertainty_2013}) have looked at tweaking these parameters to create an improvement in the simulated climate in the tropics.
\section{IF scheme}
An alternate convective scheme being developed Ian Folkins at Dalhousie University (``IF scheme") will be used in the simulations to test the sensitivity of tropical dynamics in the CAM4. This scheme was originally designed to improve the diurnal cycle of rainfall over land \citep{folkins_simple_2014}, and is now being developed to improve the simulation of the tropical temperature profile and the simulation of the MJO. One reason to desire a better tropical temperature profile is so that models can better simulate CAPE in the tropics and thus give better confidence to the simulation of tropical updrafts and downdrafts. Also, an improved static stability profile should lead to a better representation of the Hadley and Walker circulations, assuming the distribution of diabatic heating is not degraded \citep{sohn_role_2016,mitas_recent_2006}. Since this model is still under development, there is much room to tweak and play with different parameters in the model which makes it ideal for sensitivity studies. The description of the scheme in this section is based on personal communication with Ian Folkins, \cite{folkins_simple_2014} and a paper that Ian Folkins is currently writing.

There are major differences between the IF scheme and the ZM scheme. The most fundamental would be that the IF scheme uses parcels instead of plumes. Plume models have been used for many decades, and are used in schemes based off the Arakawa-Schubert entraining plume model \citep{yano_basic_2014}. However, as \cite{yano_basic_2014} points out, this model has been criticized for decades. In the IF scheme, no longer is the convective mass flux simply a function of CAPE, but also is a function of the rainfall rate of the previous timestep and the convective mass flux type is partitioned into shallow and deep (Ian Folkins, personal communication) and this removes the need for the separate shallow Hack scheme. Additionally, the CAPE used in calculating the mass flux is not simply computed by lifting a parcel from the surface to level of free convection (LFC), but the parcel properties are in fact calculated from 4 different model levels closest to the surface, then lifted to the LFC where positive buoyancy is reached, then continued until the level of neutral buoyancy is reached (LNB) \citep{folkins_simple_2014}. Another important note about this scheme is that the default cloud microphysics scheme is replaced in regions where this scheme is activated. This scheme will be described briefly in this section as well. This cloud scheme has been set-up to run from 38$^{\circ}$S-38$^{\circ}$N \textit{and} regions where CAPE calculated from the surface is $>0$.

\subsection{Updrafts}
The starting mass fluxes for a spectrum of parcels is calculated at each of the 4 lowest levels (i) and the mass per unit is defined as \citep{folkins_simple_2014}:
\begin{align}\label{eq:zmm}
M_{i} = 
     \begin{cases}
       f_{s}(P)A(P)\big(\frac{\Delta{t}}{\tau}\big)\big(\frac{\text{CAPE}_{i}}{\text{CAPE}_{scale}}\big)\big(\frac{dp}{g}\big) &\quad\text{Shallow modes}\\
      (1-f_{s})(P)A(P)\big(\frac{\Delta{t}}{\tau}\big)\big(\frac{\text{CAPE}_{i}}{\text{CAPE}_{scale}}\big)\big(\frac{dp}{g}\big) &\quad\text{Deep modes}\\
     \end{cases}
\end{align}
Where $f_{s}(P)$ is a sigmoidal function (varying from 0 to 1) of the precipitation rate of the previous time step that is used to determine the fraction of mass flux for the shallow modes with the remaining fraction the amount of mass in the deep modes. $A(P)$ is a so-called ``amplification factor", which is again a sigmoidal function that is dependent on the precipitation rate of the previous time step. $\Delta{t}$ is the time step and $\tau$ is the timescale for CAPE consumption (same as in the ZM scheme) which is set to 30 hours in this scheme. CAPE$_{i}$ is the CAPE of the specific parcel, and CAPE$_{scale}$ is set to 500 J kg$^{-1}$. $dp$ is simply the thickness of the grid box.

The sigmoidal functions, $f_{s}(P)$ and $A(P)$ are defined in the following way:
\begin{align}\label{eq:sigmoid}
A(P)=A_{min}+\frac{A_{add}}{1+e^{-P_{norm}}}
\\
f_{s}(P)=\Big(A_{min_{s}}+\frac{A_{add_{s}}}{1+e^{-P_{norm_{s}}}}\Big)
\end{align}
With $A_{min}$ and $A_{add}$ are prescribed parameters, with $P_{norm}$ defined as:
\begin{align}
P_{norm}=\frac{P-P_{half}}{P_{scale}}
\end{align}
Where $P_{half}$ is the value of $P$ for which $A(P) = A_{min} + \frac{A_{add}}{2}$ and $P_{scale}$ is a prescribed scaling factor. The parameters take on different values in the 2 formulations, with $A_{add_{s}}$ usually set to -1. This sigmoidal amplification factor approaches 1 (0 for $f_{s}(P)$) at higher rain rates, and thus $f_{s}(P)$ (the fraction of the shallow modes), approaches 0 at high rain rates, and all of the mass flux is in the deep modes (from \ref{eq:zmm}). The limit of the amplification factor is $A_{min} + A_{add}$. This method of mass flux closure method causes mass flux and rain rates to have a certain ``memory" and causes regions of strong convection to cluster and this is thought to be physically realistic (see \cite{mapes_gregarious_1993}). This leads to a tendency for convection to organize in a non-linear manner, which is not typically seen in other convective parameterization schemes (Ian Folkins, personal communication). Some examples of $A(P)$ functions for various values of the parameters are shown below:
\begin{figure}[H]
\centering
\noindent\includegraphics[width=0.8\linewidth]{../figures/amp-crop.pdf}\hfill
\caption{Example of various $A(P)$ functions with different (typical) values for the parameters. In all cases, $A_{min}$ is set to 0, and $A_{add}$ sets the top vertical asymptote. $P_{half}$ varies from 45 to 50 and $P_{scale}$ is either 25 or 30 and this sets the width of the region between the minimum of $A(P)$ and $A_{add}$. The values of $A_{add}$, $P_{half}$ and $P_{scale}$ are, 2.5, 45 mm day$^{-1}$ and 25 mm day$^{-1}$, respectively, for both versions of the CAM4-IF (orange line).}
\label{fig:sigmoid}
\end{figure}
After the starting updraft parcel mass spectra (shallow and deep) are defined \ref{eq:sigmoid} the parcel will entrain environmental air from the outside environment which reduces buoyancy until a certain target buoyancy is reached and the parcel will then completely detrain once the buoyancy is $<$ 0. If there is enough condensate in the parcel, it will begin to precipitate at any level (whereas in the ZM scheme convective precipitation is confined to above the freezing level). As the parcel rises, the parcel condensate detrains into the background atmosphere as a function of relative humidity:
\begin{align}
E_{det}=f_{det}(RH_{det}-RH)dz
\end{align}
$f_{det}$ is a prescribed parameter and if the environmental $RH$ is above $RH_{det}$, the evaporation rate is 0. In the 2 CAM4-IF versions, $RH_{det}$=0.82. For the parameterization of precipitation, first, there is a certain fraction of the updraft condensate that is removed at each level (0.24 for the deep modes and 0.30 for the shallow modes). What happens with this removed condensate then depends on the environmental relative humidity. If it is above 0.65, all of this condensate becomes precipitation, and if it is below 0.40, it is all evaporated and for values a linear interpolation to determine the fraction of evaporation to precipitation is used. If the environmental temperature at the level which detrainment occurs is below 0, and the background RH is above 0.40, then some fraction of this condensate becomes so called ``anvil snow" and this fraction $f_{an}$ is determined, again, by a sigmoidal function which depends on $P$. This falling anvil snow can then produce downdrafts when it falls through the melting level.

\subsection{Downdrafts}\label{ifdown}
Downdrafts can be produced at any level below the melting level in the CAM4-IF, which is quite different for the ZM scheme which only produces precipitation and downdrafts above the melting level. To generate a downdraft below the melting level, from either evaporating precipitation or detrained condensate, the relative humidity needs to be below a threshold level of 0.88. Similar to the ZM scheme, the local evaporation rate (kg m$^{-2}$) is in fact a function of the local precipitation rate (see \ref{eq:evap}) (Ian Folkins, personal communication):
\begin{align}
E=kq_{s}A_{RH}m_{layer}P\Delta{t}
\end{align}
Where all k is a constant which is 0.05 for rain generated from updrafts and 0.10 for downdrafts generated from anvils, $q_{s}$ is the saturation mixing ratio inside the downdraft parcel, $A_{RH}$ is another sigmoidal parameter, $m_{layer}$ is the mass of the grid cell per unit area. The other variables have the same definitions that have been defined previously. The amplification factor, $A_{RH}$ is defined as (0 for parcel RH below 0.9):
\begin{align}
A_{RH}=0.1\cdot{log}\bigg(1+\frac{0.9-RH_{parcel}}{0.1}\bigg)
\end{align} 
This factor effectively sets the dependence of the downdraft evaporation rate to relative humidity. The downdraft parcel then descends so long as its effective buoyancy, $b_{test}$, is negative. $b_{test}$ is defined in the CAM4-IF as:
\begin{align}
b_{test}=0.5(b_{i}+b_{i+1})+b_{r}
\end{align}
$b_{r}$ is a fudge factor that is meant to represent the extra negative buoyancy due to the mass of the precipitation in the parcel and is set to -0.02 m s$^{-2}$, $b_{i}$ is the parcel buoyancy at level $i$ after taking into account the evaporation at that level and $b_{i+1}$ is the buoyancy once the parcel is moved to the level below. In the CAM4-IF-r version, downdrafts below the melting level are turned off by turning up $b_{r}$ to 6 m s$^{-2}$ which effectively prevents any downdrafts that form from penetrating down to the lower levels, i.e., negative buoyancy is artificially cut off below the melting level in this version. 

Another source of downdrafts originates from the melting level, and are assumed to form from melting snow precipitating down from the anvils above. These downdrafts are parameterized in exactly the same described above for the downdrafts below the melting level, except they can only descend a maximum of 2 levels and have an extra ``boost" from the latent heat (cooling) of melting. These downdrafts are only initiated in the first level below the melting level (taken to be 273K). All the snow from above is assumed to melt and then the rain in the parcel can evaporate in the same manner as described above.
\subsection{Cloud scheme}
As mentioned previously, the cloud scheme in the CAM4 is also replaced with the IF scheme in regions equator-ward of 38$^{\circ}$N/S. The largest radiative effect is seen from the cloud ice aspect of the scheme (Ian Folkins, personal communication) and I will briefly discuss this here. There are 2 contributions to ice clouds in the new scheme; detraining ice condensate from the convective scheme, and in-situ cloud that forms in regions where the relative humidity with respect to ice is high enough (above 0.80). The vertical profile of detraining ice condensate has a gaussian shape and the mass mixing ratios of cloud ice from detrainment and in-situ formation are defined as follows:
\begin{align}
\mbox{\fontsize{17.28}{21.6}\selectfont\(
q_{ice_{det}}(z)=q_{ice_{tot}}e^{\big(-\big(\frac{z-12}{\Delta{z}}\big)^{2}\big)}
\)}
\end{align}
For the ice mass mixing ratio from detrainment, and:
\begin{align}
\mbox{\fontsize{15}{21.6}\selectfont\(
q_{ice_{in}}(z)=q_{scale}(RH_{ice}-0.80)log\big(1+\frac{q_{s_{ice}}}{q_{scale}}\big)
\)}
\end{align}
For the ice mass mixing ratio for in-situ ice cloud. Here, the  height, $z$, is in units of km, for simplicity. For heights above 12km, $\Delta{z}$=2.25km and $\Delta{z}$=4.0 km for levels below this height. $q_{ice_{tot}}$ is the total detrained ice mass mixing ratio which is then distributed vertically in the guassian profile. $RH_{ice}$ is the relative humidity \textit{with respect to ice} at the level, $z$, and $RH_{ice}$-0.80 will of course only be non-zero for a relative humidity with respect to ice above 0.80. $q_{s_{ice}}$ is the environmental saturation mixing ratio with respect to ice (which is lower than $q_{s}$) at that level and $q_{scale}=2.0\cdot10^{-5}$. 

Finally, this new ice scheme replaces the default Kristjansson scheme which relates the particle effective radius of ice particles \citep{kristjansson_impact_2000} in the CAM4 (see pg. 123-124 and figure 4.2 of \cite{neale_description_2010}) to temperature with a different relation described in \citep{garrett_small_2003}. This parameterization of the effective radius assumes smaller particles which absorb more solar radiation and this has an impact on the cloud radiative heating rate in the upper tropophere (as we will show in future sections). 
\chapter{Experimental design, methods and datasets}

\section{Experimental set-up}

In all the simulations discussed in this thesis, the finite volume (FV) dynamical core was used in the CAM4. All simulations used 26 vertical levels with a 1.9x2.5$^{\circ}$ resolution lat/lon grid. To perform the simulations, a large amount of CPUs was necessary to ensure timely completion. For this task, the GPC cluster on the SciNet system at the University of Toronto was chosen (\url{http://www.scinethpc.ca/gpc/}). Typically, the 30-year simulations were performed in under 48 hours using 64 CPUs on this cluster. 


-SST boundary conditions used: monthly mean SSTs (1982-2001 HadISST) (Hurrell et al., 2008).
\\
-1.9x2.5 degree finite volume grid with 26 vertical levels.
\\
-ocean model is turned off - atmosphere and land models only.
\\
-typically run for 30 years, CAM4 control was run for 50. 
\\
-all model versions use CAM4 from CESM1.0.2 and were run on the scinet cluster.
\\
-to test response to climate change, used a +4K SST experiment. Want to test if response to climate change is sensitive to convection scheme.
\\
-to compare directly with radiosonde dataset (1998-2011), ran CAM4 and CAM4-IF with observed monthly SSTs from 1998-2005. 
\section{Datasets}
A number of datasets were used in this thesis for comparing modern simulations to the modern day climate. 



The main goal of recent iterations of CAM-IF is to create a realistic tropical temperature profile. To do this, one needs to have a good observational dataset in the deep tropics. Since reanalysis likely has errors in the representation of the structure of tropical temperature \citep{mitas_recent_2006}, high-resolution radiosonde data is prefered for model validation. So, here we will us data from the SPARC US high-resolution radiosonde archive \citep{love_us_2013} from a number of islands in the tropics (mainly in the tropical western Pacific) for comparison to profiles of model vertical profiles of temperature. Below is a table and map of the sites for which data is available for:

\begin{table}[H]
\caption{SPARC high-resolution US tropical radionsonde sites used in this thesis}
\label{tab:rmse}
\begin{tabular}{|p{6.5cm}||p{2cm}|p{2cm}|}
\hline
Radiosonde site&Lat&Lon\\ \hline
\text{GUA - Guam}&13.55$^{\circ}$N&144.80$^{\circ}$E\\   \hline
HIL - Hilo, Hawaii&19.72$^{\circ}$N&204.93$^{\circ}$E\\ \hline
JUA - San Juan, Puerto Rico&18.43$^{\circ}$N&294.00$^{\circ}$E\\ \hline
KOR - Koror, Palau&7.33$^{\circ}$N&134.48$^{\circ}$E\\  \hline
LIH - Lihue, Hawaii&21.98$^{\circ}$N&200.65$^{\circ}$E\\  \hline
MAJ - Majuro, Marshall Islands&7.08$^{\circ}$N&171.38$^{\circ}$E\\  \hline
\text{PAG - Pago Pago, American Samoa}&14.33$^{\circ}$S&189.28$^{\circ}$E\\  \hline
PON - Pohnpei, Micronesia&6.97$^{\circ}$N&158.22$^{\circ}$E\\  \hline
TRU - Truk Lagoon, Micronesia&7.47$^{\circ}$N&151.85$^{\circ}$E\\  \hline
\end{tabular}
\label{tab:sites}
\end{table}

\begin{figure}[H]
\centering
\noindent\includegraphics[width=0.3\linewidth, angle=90]{../figures/stations.pdf}\hfill
\caption{Map of tropical radiosonde sites from the SPARC US high-resolution radiosonde archive that are indicated in table \ref{tab:sites}.}
\label{fig:stations}
\end{figure}

However, the radiosonde dataset is quite limited spatially, and although temperature variations in the tropics are small (see section \ref{wtg}), to gauge tropical mean quantities, a more continuous dataset is needed. So we will use reanalysis data for comparison of simulations to modern climate as well. There are a variety of reanalysis datasets available, however, the MERRA (from NASA) and ERA-Interim (from ECMWF) reanalyses are thought to have the most accurate representation of the climate since 1979. Also, we have climate model simulations from the CMIP5 AMIP (Atmospheric Model Intercomparison Project) model suite \citep{taylor_overview_2011} that we can use to compare with our climate simulations of the modern climate (AMIP models are atmosphere-only runs like the ones we have performed here). Combined, the radiosonde data, reanalysis data and other GCM model data can help gauge if the IF scheme is improving on, or degrading the modern tropical climate in the CAM4 model compared to observations and other models.

\chapter{Uniform SST increase simulations}
\section{Control simulations - modern day climate}
Before examining the differences in the +4K SST response of the CAM4-IF compared to the CAM4, it was prudent to perform sanity check first to see how the model compares to the performance of the default CAM4 in modern climate. If a strong shift away from observations was seen in the simulation of the overall climate and in the variables we are interested in was seen, it could call into question any conclusions made about this new models' response to climate change. Even if these were the case, however, it could still prove to be a helpful idealized model in understanding tropical dynamics. 

In this section we will show that some definite changes were seen, but these were deemed acceptable, with the tropical variables we are most interested in (temperature, precipitation and winds), comparable to or in fact better simulated compared to observations than the CAM4 with the ZM scheme. Since most of the dynamics in the tropics are driven by convection as baroclinic instability is almost entire absent due to small temperature gradients, any change in the scheme could be expected to produce differences in the simulation of tropical phenomenon (but perhaps may not change tropical mean quantities, though). The biggest issue with the IF scheme appears to be an overactive South Asian Summer Monsoon (SASM). In this section, we will compare the default CAM4 to 2 versions of the CAM4-IF run for different variables in the tropics using modern SSTs (F\_2000 compset - see previous section for more details about the simulation set-up). We will also briefly discuss the performance of the simulation of various tropical atmospheric waves such as Kelvin waves and the MJO. 

\subsection{Temperature}

Although it is often assumed the tropical mean temperature profile is essentially moist adiabatic \citep{emanuel_quasi-equilibrium_2007}, there are some small deviations, such as around the melting level \citep{folkins_melting_2013}. These small deviations from the moist adiabat are what the CAM4-IF mainly seeks to improve upon. This would lead to an improvement of the lapse rate ($\frac{\partial{T}}{\partial{z}}$) profile as well. 
The CAM4-IF T profile in the CAM4-IF-t version is much improved over the CAM4 in the deep tropical convecting regions, with the CAM-IF having a lower pressure-weighted RMSE at most SPARC deep tropics sites than all the AMIP model members (see figure \ref{fig:3.2} \textbf{a)} for an example at a location in the tropical western Pacific: Koror, Palau). This site is particularly important because it is located in the western Pacific warm pool, where significant time-mean rainfall occurs. The CAM4-IF-r has a slight degradation in the tropical T profile, especially near the surface due to the absence of lower tropospheric downdrafts (not shown). Generally, most models have a lower tropospheric cold bias and some models have an even larger upper tropospheric cold bias (max around 850 hPa), while the CAM4 has an mid tropospheric warm bias (max around 500 hPa). See table \ref{tab:rmse} for RMSEs for 8 other radiosondes sites in the deep tropics.

I discussed in a section 1.2.1 about how the tropical mean static stability is important for the strength of the tropical circulations, especially the Walker circulation. Since static stability (integrated through the troposphere) is determined by $\frac{\partial{T}}{\partial{z}}$, ensuring the vertical profile of T is as close to observations as possible in a climate model should help in the correct representation of the strength of the tropical circulations in a climate model (all else being equal) \citep{sohn_role_2016,mitas_recent_2006}. It would appear that the CAM-IF is fairly successful at improving the T profile over the default CAM, and is even better than all the AMIP members at many radiosonde sites in the deep tropics. 


\begin{figure}[H]
\centering
\noindent\includegraphics[width=1\linewidth]{../figures/T_figure-crop.pdf}\hfill
\caption{\textbf{a}: Annual mean Koror model minus observed temperature profile. \textbf{b}: Annual mean Koror lapse rate. \textbf{c}: Annual mean 20$^\circ$S-20$^\circ$N mean lapse rate. Overall, the CAM4-IF-t improves upon the CAM4 and is better than the AMIP mean in terms of the lapse rate profile throughout the deep tropics, particularly with the low-level stability maximum near 850 hPa. Note the y-axis scale is logarithmic in all 3 panels.}
\label{fig:3.2}
\end{figure}
\begin{table}
\caption{Pressure-weighted RMSEs for various SPARC high-resolution radiosonde sites for the CAM4, CAM4-IF-t and the AMIP mean for 1998-2005 (all AMIP-style simulations). The rank of the CAM4-IF compared to the CAM4, and the 20 AMIP members is indicated in the right column, with a rank of ``1" indicating that the CAM4-IF-t has the lowest pressure-weighted RMSE of all models. The CAM4-IF-t has the lowest pressure-weighted RMSE at most locations in this dataset. Note, the radiosonde dataset is from 1998-2011, but there is only data up to 2005 available for all AMIP models and the CAM4 only has complete boundary conditions to run until the end of 2005 (for CESM1.0.2).}
\label{tab:rmse}
\begin{tabular}{|p{2cm}||p{2cm}|p{2.25cm}|p{2.25cm}|p{2.25cm}|p{3cm}|}
\hline
\multicolumn{6}{|c|}{Temperature RMSE}\\
\hline
Site&CAM4&CAM4-IF-t&CAM4-IF-r&AMIP mean&\text{CAM4-IF-t rank}\\ \hline
\text{GUA}&0.454&0.419&0.349&0.559&1\\   \hline
HIL&0.513&0.500&0.486&0.515&2\\ \hline
JUA&0.411&0.373&0.290&0.445&1\\ \hline
KOR&0.423&0.314&0.423&0.571&1\\  \hline
LIH&0.574&0.415&0.415&0.632&4\\  \hline
MAJ&0.257&0.227&0.255&0.534&1\\  \hline
\text{PAG}&0.648&0.609&0.520&0.674&1\\  \hline
PON&0.452&0.393&0.446&0.604&1\\  \hline
TRU&0.450&0.376&0.408&0.618&1\\  \hline
\end{tabular}
\label{tab:sites}
\end{table}
\subsection{Precipitation}
To demonstrate that using this different convection/cloud scheme with the CAM4 does not cause a major degradation in the representation of modern climate, here we will examine the performance of the CAM4 and CAM4-IF using reanalysis and observations (satellite and weather balloon soundings) as a benchmark. First we start with time-mean tropical precipitation (figure \ref{fig:3.1} and table \ref{tab:1}). We see that the CAM4-IF-r version and the CAM4 have similar tropical mean rainfall RMSEs, with the CAM4-IF having a better pattern correlation with the observational dataset. The 2 CAM4-IF models  have less of a ``double-ITCZ" rainfall pattern than the CAM4 and are closer to observations which can be seen in the zonal mean charts on the right of each plot. The CAM4-IF models tends to have too much precipitation in the SASM region around 15$^\circ$N from India to the Philippines, although this is reduced to some extent in the b.r. run. All 3 models have a higher tropical mean rainfall rate than the blended observations, although the mean values are close to those seen in reanalysis data (the satellite data included in this ``blend" bring down the tropical mean). It has been suggested that rainfall in the GPCP precipitation product (and other products derived from satellite data) is likely biased too low when the global mean energy budget is considered \citep{trenberth_earths_2009,trenberth_regional_2013}.

The CAM4-IF-t has a better representation than both the CAM4 and the other CAM4-IF version for tropical mean land rainfall, with ocean rainfall RMSE the lowest in the CAM4-IF-r version in the annual mean and JJA. This could indicate that penetrative downdrafts which reach the lower troposphere is important for simulating land rainfall, although they appear to lead to an over-active summer monsoon. Overall, with the reduction of the double-ITCZ bias seen and better tropical mean pattern correlation seen in the CAM4-IF-r, it is likely that this version has the best representation of the tropical circulations (HC and WC), as these are what structure the distribution of rainfall in the tropics.
\begin{table}
\caption{Tropical mean (30$^{\circ}$S-30$^{\circ}$N) annual mean total precipitation rate ($P$) RMSE for the 3 CAM4 models and the 12 AMIP modern control members. See section x.x for more information about the datasets and models included here. While the three CAM4-IF model RMSEs' are on the high side of the AMIP model suite, the tropical mean precipitation is near the median of the AMIP members of 3.54 mm day$^{-1}$.}
\label{tab:rmse}
\begin{tabular}{|p{4.4cm}||p{1.25cm}|p{1.25cm}|p{1.25cm}|p{1cm}|p{1.5cm}|p{2.9cm}|}
\hline
\multicolumn{6}{|c|}{$P$ RMSE (mm day$^{-1}$)}&\\
\hline
Model&CMAP&GPCP&TRMM&ERAI&MERRA&30$^{\circ}$S-30$^{\circ}$N mean\\ \hline
CAM4 default&1.41&1.44&1.55&1.34&1.41&3.51\\   \hline
CAM4-IF-r&1.44&1.45&1.53&1.38&1.36&3.47\\ \hline
CAM4-IF-t&1.56&1.67&1.74&1.56&1.55&3.51\\ \hline
\text{AMIP - bcc-csm}&1.45&1.46&1.53&1.45&1.43&3.27\\  \hline
\text{AMIP - CanAM4}&1.49&1.66&1.70&1.34&1.51&3.40\\  \hline
\text{AMIP - CCSM4}&1.30&1.33&1.44&1.12&1.23&3.61\\  \hline
\text{AMIP - CESM1-CAM5}&1.28&1.38&1.50&1.05&1.20&3.72\\  \hline
\text{AMIP - CNRM-CM5}&1.21&1.54&1.59&1.17&1.33&3.66\\  \hline
\text{AMIP - FGOALS-g2}&2.17&2.25&2.31&2.13&2.07&3.52\\  \hline
\text{AMIP - IPSL-CM5A-LR}&1.27&1.30&1.38&1.30&1.34&3.29\\  \hline
AMIP - IPSL-CM5B-LR&1.25&1.37&1.38&1.36&1.24&3.28\\  \hline
AMIP - MIROC5&1.71&1.74&1.91&1.59&1.65&3.93\\  \hline
AMIP - MPI-ESM-LR&1.13&1.43&1.46&1.08&1.27&3.55\\  \hline
AMIP - MPI-ESM-MR&1.15&1.46&1.50&1.09&1.31&3.58\\  \hline
AMIP - MRI-CGCM3&1.17&1.38&1.41&1.19&1.16&3.54\\  \hline
\end{tabular}
\label{tab:P}
\end{table}
\begin{figure}[H]
\centering
\noindent\includegraphics[height=0.8\linewidth]{../figures/ANN_pr.pdf}\hfill
\caption{Annual mean precipitation rate for \textbf{a}: the default CAM4, \textbf{b}: the CAM4 with the IF scheme that produces a tropical mean annual mean rainfall distribution with the lowest RMSE when compared to reanalysis, \textbf{c}: the CAM4 with the IF scheme that produces tropical temperature profiles closest to those observed from radiosondes. In \textbf{b and c}, stippling indicates regions where the difference between the CAM4-IF and CAM4 default rainfall is greater than 1 mm day$^{-1}$ above the standard deviation of the observational blend. \textbf{d}: Observational rainfall blend mean (see table x.x), with stippling indicating regions where the standard deviation of the dataset blend is $>$ 1 mm day$^{-1}$. Here, and for the rest of this thesis, r$_{pat}$ indicates the pattern (spatial) correlation coefficient. The annual mean tropical mean precipitation rate ranges from 3.27 mm day$^{-1}$ to 3.93 mm day$^{-1}$  for the AMIP modern models., providing support for the higher tropical mean amounts seen in the CAM4 models.}
\label{fig:3.1}
\end{figure}

\begin{table}[H]
\caption {Seasonal and regional rainfall table. All RMSEs are a comparison with the observational blend described in section .x.x. } \label{tab:title} 
\begin{center}

\begin{tabular}{|p{4cm}||p{3cm}|p{2cm}|p{2cm}|  }
\hline
\multicolumn{4}{|c|}{ANN Rainfall RMSE (mm/day)}\\
\hline
Model&Region&30$^\circ$S-30$^\circ$N&15$^\circ$S-15$^\circ$N\\    \hline
Default CAM4&Land&1.66&1.78\\    \cline{2-4}
&Ocean&1.05&1.29\\    \hline
\text{CAM4-IF-r}&Land&1.73&2.19\\   \cline{2-4}
&Ocean&0.99&1.21\\   \hline
CAM4-IF-t&Land&1.56&1.77\\   \cline{2-4}
&Ocean&1.43&1.60\\   \hline
\end{tabular}

\begin{tabular}{|p{4cm}||p{3cm}|p{2cm}|p{2cm}|  }
\hline
\multicolumn{4}{|c|}{DJF Rainfall RMSE (mm/day)}\\
\hline
Model&Region&30$^\circ$S-30$^\circ$N&15$^\circ$S-15$^\circ$N\\    \hline
Default CAM4&Land&2.29&2.67\\    \cline{2-4}
&Ocean&1.74&2.30\\    \hline
\text{CAM4-IF-r}&Land&2.44&3.14\\   \cline{2-4}
&Ocean&1.90&2.31\\   \hline
CAM4-IF-t&Land&2.06&2.58\\   \cline{2-4}
&Ocean&2.00&2.26\\   \hline
\end{tabular}

\begin{tabular}{|p{4cm}||p{3cm}|p{2cm}|p{2cm}|  }
\hline
\multicolumn{4}{|c|}{JJA Rainfall RMSE (mm/day)}\\
\hline
Model&Region&30$^\circ$S-30$^\circ$N&15$^\circ$S-15$^\circ$N\\    \hline
Default CAM4&Land&2.94&2.38\\    \cline{2-4}
&Ocean&2.10&2.49\\    \hline
\text{CAM4-IF-r}&Land&2.67&2.95\\   \cline{2-4}
&Ocean&1.90&2.34\\   \hline
CAM4-IF-tT&Land&2.79&2.70\\   \cline{2-4}
&Ocean&2.62&2.74\\   \hline
\end{tabular}
\end{center}
\label{tab:1}
\end{table}




\subsection{Tropical Dynamics}
The main focus of this thesis is on the tropical circulations so while the CAM4-IF has an improved temperature profile and tropical precipitation that is at least comparable in accuracy to the CAM4 in the modern climate, it is not guarantee that the representation of these circulations is not degraded by using switching to the IF scheme. Probably the best variable to check would be 500 hPa $\omega$, which is often used as a proxy for circulation strength as this is typically around the level at which vertical motions in the troposphere reach their maximum \citep{vecchi_global_2007}. For the CAM models, the 3d output variables on sigma levels, so to get the 500 hPa $\omega$ (hereafter, $\omega_{500}$) we took the sigma level where the pressure level is usually closest to the 500 hPa level, sigma level 18.

Overall, the CAM4-IF has errors of similar magnitude for $\omega_{500}$ as the CAM4 using the ZM scheme when compared to a blend of the ERA-Interim and MERRA reanalysis datasets (figure \ref{fig:omega}). The RMSEs for the two CAM4-IF versions and the CAM4 are quite close, with the CAM4 b.r. having the best pattern correlation, and since the $\omega_{500}$ is very similar to the $P$ field spatially, this is consistent with the fact that the CAM4-IF-r has the best pattern correlation for $P$ (see figure \ref{fig:3.1}). The double-ITCZ pattern in $\omega_{500}$ is still apparent in the zonal mean plot for the CAM4, with the CAM4-IF models slightly improved compared to an ERA-Interim/MERRA reanalysis blend in panel \textbf{d)}. All three models appear to have a poor representation of $\omega_{500}$ over the Indian Ocean, similar to what was seen with $P$. The tropical maximum in $\omega_{500}$ is over Papua New Guinea, consistent with what is seen in all three models. Overall, the CAM4-IF-r has some improvements over CAM4, but the improvements are not as great as one would expect from the much improved tropical T profile.
\begin{figure}[H]
\centering
\noindent\includegraphics[width=0.80\linewidth, angle=90]{../figures/ANN_omega500.pdf}\hfill
\caption{ANN climo ~500 hPa omega (sigma level 18) with comparisons to a MERRA/ERAI reanalysis blend.}\
\label{fig:omega}
\end{figure}

Another variable that is often used to gauge the strength and spatial pattern of the tropical circulation is velocity potential, $\chi$, in the upper troposphere, typically 200 hPa (see \ref{eq:chi} and \cite{tanaka_trend_2004}). Since we are mainly focusing on the zonal (Walker circulation) component of the tropical circulation, we will remove the zonal mean to remove the zonally uniform component (Hadley circulation). We call this velocity potential with the zonal mean removed $\chi^{*}$. Figure \ref{fig:chistar} shows the annual mean 200 hPa $\chi^{*}$ for the three different CAM4 versions, and it is clear that the CAM4-IF versions have a spatial pattern closer to that of reanalysis, especially for the divergent region associated with ascending region of the WC (negative $\chi^{*}$). The overall tropical mean RMSE is lowest in the CAM4-IF-r, consistent with its performance with $P$ and $\omega_{500}$. The CAM4 appears to have too much divergence over central Africa compared to reanalysis, with the CAM4-IF versions having convergence in that region. The magnitude of the divergence in the ascent region of the WC in the CAM4-IF versions is too strong compared with reanalysis.
\begin{figure}[H]
\centering
\noindent\includegraphics[width=0.66\linewidth, angle=90]{../figures/200hpachistar_ANN.pdf}\hfill
\caption{ANN climo 200 hPa $\chi^{*}$ (shaded) with an ERAI/MERRA reanalysis blend overlaid as contours with a spacing of $10^{6}$ $m^{2}$/$s$ for both contours and shading.}
\label{fig:chistar}
\end{figure}
\begin{figure}[H]
\centering
\noindent\includegraphics[height=1\linewidth, angle=90]{../figures/ANN_M_omega.pdf}\hfill
\caption{$M_{int}$ - pressure-thickness weighted vertically integrated mass flux from 1000-100 hPa with the pattern correlation between $M_{int}$ and the pressure-thickness weighted vertically integrated integrated omega for the same interval indicated. Here the sign convention for upward mass flux is taken to be negative similar to that of $\omega$.}
\label{fig:mcomeg}
\end{figure}
Overall, it appears that the CAM4-IF version of the CAM4 improves on some significant features in the tropics, the tropical temperature and lapse rate profile and the tropical circulation. There is a deterioration in the precipitation representation in the versions of the CAM4 using the IF scheme, and this is mainly seen in the SASM region and the Indian Ocean. However, none of these variables are directly influenced by the convective scheme per se, and are just a response to changes in the distribution of heating due to convection which presumably will change somewhat due to differences in the convective mass flux model used. subgrid scale mass flux (net of updrafts + downdrafts) is a variable that we can examine in all 3 model versions. We can see immediately from figure \ref{fig:mcomeg} that the CAM4 vertically integrated tropical mean mass flux (hereafter, $M_{int}$) is larger than in the CAM4-IF versions.  So in other words, while the CAM4 default and CAM4-IF versions have comparable tropical mean levels of precipitation, the CAM4 has significantly more upward convective mass flux and thus a smaller amount of precipitation produced per unit of mass flux. The spatial correlation of $M_{int}$ with the vertically integrated $\omega$ is quite high and thus these two quantities and $P$ (which itself is highly correlated with negative $\omega$), are spatially closely related. This is because the strength of the mass flux controls the diabatic heating rate from condensational heating and this diabatic heating controls $\omega$. 


\section{Response to a uniform 4K SST warming}

\subsection{General overview}

The main forcing that causes the tropical circulations to weaken under climate change is thought to be the mean SST warming \citep{he_anthropogenic_2015}. Thus, this thesis focuses on the response of the CAM4-IF to a global mean 4K SST warming (see methods for more details). This experiment was performed with the IPCC models, and is referred to as the ``AMIP4K" experiment. Here we will mainly focus on examination of time-mean quantities such as annual means. We find that the while the tropical circulations do weaken in the CAM4-IF, the tropical mean convective mass flux does not weaken as theory would suggest. This suggests that the hydrological cycle argument to the weakening of tropical mean convective mass flux may be sensitive to the type of convective scheme used.

The precipitation rate response to warming is constrained by the increase in downward longwave radiative flux (see section 1.1.1). In response to a uniform 4K SST increase, the atmosphere warms so as the planet can regain radiative balance, and thus the downward longwave radiation increases (the atmosphere increases radiation isotropically by the Stefan-Boltzmann law). The global mean increase in downward longwave radiation for the CAM4 and the various CAM4-IF versions is found to be in a tight range 30-31 Wm$^{-2}$, while the global mean surface temperature increases fairly uniformly by around 4.3-4.4K. The global mean tropospheric temperature increases for than 4K because of the increased condensational heating to the very large increase in water vapour capacity of the atmosphere due to the C-C relation. The tropical (and global) mean precipitation response among the CAM4 and CAM4-IF models has some variation and is in the range of $\approx$ 16-19\% which implies an increase of $\approx$ 3.5-4.5\% K$^{-1}$ (see table \ref{tab:title}).

Similar to the results of \citep{vecchi_global_2007}, the decrease in tropical mean $\omega_{500}^{\uparrow}$=$\omega_{500}{H(-\omega_{500})}$, that is, upward vertical motion only at the 500 hPa level, (using the notation from \cite{schneider_water_2010} and \cite{merlis_changes_2011}) is generally less, and in some case half of the inferred mass flux decrease based on the hydrological cycle. The correlation coefficient between the inferred mass flux response and the tropical mean $\omega_{500}^{\uparrow}$ mass flux between the 10 model runs performed here is 0.57, indicating that the relationship is not particularly robust, however, the models are certainly not independent. Regardless, the fact that the upward omega decreases less than inferred raises the possibility that omega may not be directly related to the convective mass flux. Also, the relationship between the $\omega_{500}^{\uparrow}$ response and inferred mass flux response is not one-to-one in figure \ref{fig:vecchi} and this is also seen in figure 4b) in \citep{vecchi_global_2007} for the CMIP5 models. This could indicate that the hydrological cycle argument is not the main reason behind the weakening of the tropical circulations due to climate change.

\begin{table}[H]
\caption {Precipitation, $q_{bl}$, inferred convective mass flux and upward 500 hPa omega response. All quantities indicated here are tropical means (30$^\circ$S-30$^\circ$N). Here, $q_{bl}$ is the mean $q$ of the 3 lowest model levels.} \label{tab:title} 
\begin{center}

\begin{tabular}{|p{4.5cm}||p{1.25cm}|p{1.5cm}|p{1.75cm}|p{2.5cm}|p{2.25cm}|  }
\hline
\multicolumn{6}{|c|}{+4K SST response}\\
\hline
Model&$\delta{P}/{P}$&$\delta{(q_{bl})}/q_{bl}$&$\frac{\delta{P}}{P}-\frac{\delta{(q_{bl})}}{q_{bl}}$&$\frac{\delta{P}}{P}-0.07(\Delta{T})$&$(\delta{\omega_{500}^{\uparrow}})/\omega_{500}^{\uparrow}$\\    \hline
1) Default CAM4&0.161&0.302&-0.142&-0.130&-0.054\\   \hline
\text{2) CAM4-IF-r}&0.192&0.281&-0.089&-0.093&-0.041\\ \hline
\text{3) CAM4-IF best T}&0.185&0.285&-0.099&-0.103&-0.068\\ \hline
4) CAM4-IF 1&0.190&0.289&-0.097&-0.112&-0.068\\  \hline
5) CAM4-IF 2&0.177&0.274&-0.095&-0.099&-0.061\\  \hline
6) CAM4-IF 3&0.190&0.285&-0.116&-0.119&-0.041\\  \hline
7) CAM4-IF 4&0.177&0.293&-0.112&-0.120&-0.091\\  \hline
8) CAM4-IF 5&0.176&0.288&-0.092&-0.096&-0.096\\  \hline
9) CAM4-IF 6&0.192&0.285&-0.135&-0.138&-0.033\\  \hline
10) CAM4-IF 7&0.159&0.294&-0.101&-0.108&-0.104\\  \hline
\end{tabular}
\label{tab:overview}
\end{center}
\end{table}
\begin{figure}[H]
\centering
\noindent\includegraphics[height=0.8\linewidth]{../figures/Rplots.pdf}\hfill
\caption{Response of tropical mean upward 500 hPa omega (upward mass flux) in various versions of the CAM4-IF. Model ``1" is the default CAM4 (the numbered model versions in table \ref{tab:overview} correspond to the numbers in this plot). Note that upward omega is not the same as convective mass flux. The dashed line is the linear least-squares fit, with the solid line the line indicating what the relationship would be if there was a  one-to-one relationship between inferred mass flux and ${\omega_{500}^{\uparrow}}$.}
\label{fig:vecchi}
\end{figure}

\begin{figure}[H]
\centering
\noindent\includegraphics[width=0.9\linewidth, angle=90]{../figures/ANN_pr_response.pdf}\hfill
\caption{\textbf{a)-d)}: ANN precipitation rate response from a 4K SST warming (shaded) with control run climatology (2 mm/day contours, starting at 4 mm/day). In \textbf{a)-c)}: there is stippling and hatching to indicate regions where the precipitation response is above the maximum AMIP4K members' response (stippling) or vice versa (hatching). The stippling (for positive responses) and hatching (for negative responses) in \textbf{d)} indicates regions where 10 or more of 12 models agree on the sign of the response. The tropical mean percentage responses are indicated. The range of tropical mean increases in precipitation in the AMIP4K models is $\approx$ 10-16\%, so the CAM4 model responses here exceed the upper range of tropical mean precipitation responses seen in the CMIP5 AMIP models.}
\label{fig:presponse}
\end{figure}
The spatial response in $P$ among the 3 CAM4 models is fairly similar, with an expansion poleward of the ITCZs seen (figure \ref{fig:presponse}). The overall spatial pattern of the $P$ response of the 3 models is also similar to that of the CMIP5 AMIP ensemble mean (panel \textbf{d)}), but is larger in magnitude. In fact, the 3 CAM4 models have large regions over the tropical oceans where the response is above the maximum response seen in the 12 AMIP models (as indicated by stippling). The general response of tropical rainfall to this uniform 4K increase could be described as an expansion of the wet regions poleward and an overall enhancement of the wet regions, with some regions near the equator seeing a decrease. There is a good agreement in a reduction or little change in rainfall in regions of the Maritime Continent near the equator, regions in which see the largest amounts of precipitation in today's climate. There are also reductions in precipitation seen in various regions of Africa in the 3 CAM models and a robust regions of decrease (indicated by hatching) seen in the AMIP mean. In fact, in the CAM4-IF models there is a notable decrease in precipitation over many land regions in the tropics, which is not as apparent in the CAM4, with increases over most ocean regions, except along the equator. This would indicate that rainfall over land in the CAM4-IF may be being suppressed by to compensate for the increases over the ocean. This would tend to indicate a weakening of the local monsoon/seabreeze circulations. The overall response is fairly consistent with the ``rich-get-richer" pattern of tropical precipitation change \citep{held_robust_2006,chou_evaluating_2009}.

It should be very clearly noted that a uniform increase in SST is not the only effect of climate change that will influence tropical precipitation. These experiments here are using atmosphere only models only and do not take into account the effects of changing patterns of tropical SST and the radiative effects of CO2 on the temperature structure of the tropical troposphere. These experiments should not be seen as equivalent to the RCP8.5 fully coupled climate model simulations. A good outline of the relative importance of the mean SST, CO2 and SST pattern effects is done in \citep{he_anthropogenic_2015}. The mean SST effect is thought to be most important for the overall weakening of convective mass flux and the tropical circulations in the tropics, however \citep{held_robust_2006,ma_mechanisms_2011,vecchi_global_2007,he_anthropogenic_2015}. Furthermore, non-linear interactions between the atmosphere and ocean are not taken into account in these atmosphere-only simulations.

While the overall precipitation response in the CAM4-IF is somewhat consistent with the default CAM4 and the AMIP mean, albeit with a larger tropical mean response, this brings us back to the question of what is the response of the tropical mean mass flux? As we can see from table \ref{tab:title}, using relation \ref{eq:HS} would imply a weakening of the convective mass flux in the range of 9-14\% in the 3 CAM4 models. However, as figure \ref{fig:vecchi} shows, the relationship between the $\omega$ response and $M^{`}$, the inferred mass flux, is not one-to-one, with $\omega$ weakening less than what would be expected from the inferred mass flux and the correlation between the 2 only 0.57. We will show in the next section that in fact the tropical mean convective mass flux in the CAM4-IF models \textit{increases} even while the tropical circulations weaken, in contrast to the CAM4 default which shows both the tropical mean mass flux and tropical circulations weakening. Even in the CAM4 default, the tropical mean ascent response as measured by $\omega_{500}^{\uparrow}$ is less than half that predicted using the weakened mass flux argument (see column 6 of table \ref{tab:title}). This hints that the inferred mass flux may not be a great estimate of the true convective mass flux, or that $\omega$ and mass flux may not be as closely linked as expected.
\subsection{Convective mass flux response}
While the difference between the fractional response of tropical mean $q_{bl}$ and $P$ would seem to imply that the tropical mean mass flux \textit{must} go down, no matter what model of convective scheme is used, our calculations of the real tropical mean mass flux for the CAM4-IF models shows this is not the case. Here, we have used the methodology of \citep{chadwick_spatial_2012} to calculate a vertically integrated (1000-100 hPa) tropical mean mass flux, $M_{int}$. Since the mass flux would produce precipitation at any tropospheric level with upward convective mass flux, and the precipitation is used to provide an estimate of mass flux, $M^{`}$ can be thought as being a measure of $M_{int}$. This is likely to be a better measure then just using the mass flux at an individual level, and indeed as we will show, the responses of $M_{c}$ can be different at different vertical levels. Figure \ref{fig:mcomegres} shows the spatial response of $M_{int}$ and vertically integrated $\omega$ ($\omega_{int}$), there is a strong spatial relationship in the response of these 2 variables, with the correlation ranging from 0.75 to 0.87 among the 3 models.

The 30$^{\circ}$S-30$^{\circ}$N mean response of $M_{int}$ (figure \ref{fig:Mc}) shows some very interesting and surprising features. We find that the mass flux does indeed weaken in the CAM4 (figure \ref{fig:Mc}\textbf{a}), as expected, although by much less than the $\approx$ 14\% predicted. However, $M_{int}$ does not weaken in the CAM4-IF models, and in fact increases by $\approx$ 16\% in the CAM4-IF-r! There is also very interesting vertical structure in the response, with an opposite-signed response in mass flux below 600 hPa between the default CAM4 and the CAM4-IF models. The CAM4 default sees a strong increase around 500 hPa, likely due to the minimum seen in the climatology (likely associated with the melting level downdrafts) moving upwards in the warming climate. This highlights why picking a single level to calculate the $M_{C}$ response at is not an accurate reflection of the vertically integrated response. Where all 3 models do agree is the upward expansion of the convective mass flux as the convection depth increases and decreases in $M_{int}$ over Africa (which corresponds nicely with the $P$ response shown in figure \ref{fig:presponse}. The second column of \ref{fig:Mc} shows that the structure of the $\omega$ response is quite different from that of $M_{c}$, which clearly demonstrates that $\omega$ and $M_{c}$ can have different responses and are only loosely related. This is the case for all 3 models, with the exception being over the land areas of Africa and South America (also see figure \ref{fig:mcomegres}).

\begin{figure}[H]
\centering
\noindent\includegraphics[width=0.75\linewidth, angle=90]{../figures/ANN_M_omega_response.pdf}\hfill
\caption{\textbf{a-c}: As in figure \ref{fig:mcomeg}, but for responses and with $\omega_{int}$ indicated in contours, with negative indicating a negative (increased upward motion) response, and solid a positive one.}
\label{fig:mcomegres}
\end{figure}
\begin{figure}[H]
\centering
\noindent\includegraphics[width=1\linewidth]{../figures/FMASS_ANN-crop.pdf}\hfill
\caption{\textbf{a, c, e}: response of the annual mean 30$^\circ$S-30$^\circ$N mean convective mass flux for the 2 CAM4-IF models and the default CAM4 with the same plots but for omega in the second column (\textbf{b, d, f}). A negative sign convention is used here for the convective mass flux to indicate upward motion, similar to $\omega$. Negative contours are solid and positive contours are dashed. Note the positive response in tropical mean vertically integrated $M_c$ ($M_{int}$) seen in the CAM4-IF models compared to the negative response predicted in column 4 \& 5 of \ref{tab:overview} and the negative response in tropical mean ascent seen in column 6 of the same table. }
\label{fig:Mc}
\end{figure}
The vertical structure of $M_{c}$ in the control runs and in the response is different between the CAM4-IF and the CAM4 default. Figure \ref{fig:mprofs} shows this clearly, with the CAM4 default having a more ``bottom-heavy" profile with a maximum of mass flux in the lower troposphere, while the CAM4-IF models having a smaller maximum in the lower troposphere. The maximum in the lower troposphere in the CAM4 is apporimxately a factor of 3 larger than that of the CAM4-IF models and a tropical mean $M_{int}$ apporximately a factor of 2 greater than that of the CAM4-IF models. Comparing these 3 profiles to the profiles in the CMIP5 models (figure 2 in \cite{chadwick_spatial_2012}), the CAM4-IF models are most qualitatively similar to the HadGEM2-ES model. It is very clear that the net convective mass flux increases in both CMA4-IF versions in the +4K warming runs, except for a decrease between 500 and 600 hPa associated with an upward shift in the melting level minimum.
\begin{figure}[H]
\centering
\noindent\includegraphics[width=1.1\linewidth]{../figures/M_profile-crop.pdf}\hfill
\caption{\textbf{a-c}: vertical profiles of the annual mean tropical mean \textit{total} convective mass flux response and control run climatology. Note the vertical scale is not logarithmic as in previous figures to highlight the "bottom-heavy" nature of the profiles.}
\label{fig:mprofs}
\end{figure}
Examining the updraft and downdraft components of $M_{c}$ in the CAM4-IF models (figure \ref{fig:updown}) shows that both versions actually have a qualitatively similar vertical profile of updraft mass flux response of 0-10$^{-3}$ kg m$^{-2}$ s$^{-1}$ (\textbf{c-d}). However, the difference in downdraft mass flux in the control and response between the 2 CAM4-IF models is quite revealing. The CAM4-IF-r has penetrative downdrafts turned off (see \ref{ifdown} for more information on the downdraft parameterization in the IF scheme), so downdrafts originating from below the melting layer do not exist. So while the CAM4-IF-t control version has a maximum of $M_{d}$ below 850 hPa, the CAM4-IF-r version has a maximum between 600-700 hPa. The freezing level maximum in both models shifts upward in response to the uniform 4K SST warming with no significant changes in the magnitude of this maximum. On the other hand, $M_{d}$ increases below the melting level in the CAM4-IF-t, with the maximum staying at the same level. This asymmetry in $M_{d}$ explains why the $M_{c}$ response in the lower troposphere is less in the CAM4-IF-t than in the CAM4-IF-r even though the CAM4-IF-t has a larger increase in $M_{u}$ in the lower troposphere. 
\begin{figure}[H]
\centering
\noindent\includegraphics[height=0.72\linewidth]{../figures/M_profile_updown-crop.pdf}\hfill
\caption{ \textbf{a) and b)}: Vertical profiles of the annual mean tropical mean updraft mass flux, $M_{u}$, for the CAM4-IF models. \textbf{c) and d)}: Downdraft mass flux, $M_{d}$, for the CAM4-IF models. Note that the updraft mass flux increases by approximately the same amount in both models versions, but downdraft strength increases more in the CAM4-IF-t CAM4-IF version.}
\label{fig:updown}
\end{figure}
As we have shown previously in figure \ref{fig:mcomegres}, the response of $\omega_{int}$ and $M_{int}$ are strongly spatially correlated, so a different signed response in the tropical mean $\omega$ and mass flux responses in the CAM4-IF models would seem to imply that the relationship between these 2 variables is changing under climate change, even though they remain very closely related in their spatial patterns. Also, the sign of the tropical mean repsonse in $M^{`}$ is opposite to that of $M_{int}$ in the CAM4-IF models, so a good start would be to examine how well the relationship between $M^{`}$ and $M_{int}$ is at each gridpoint; this is done is figure \ref{fig:pe} \textbf{a-c}. We can see that the there is a fairly strong correlation between $M_{int}$ and $M^{`}$  on gridpoint-by-gridpoint basis in all 3 models, however the slope is much steeper than 1 for the CAM4 default, indicating the efficiency of $M_{int}$ (even for regions with $P$ greater than 4 mm day$^{-1}$) to produce precipitation is lower than what would be predicted by theory. However, the slopes and y-intercepts of the linear regression fit line do appear to change with the 4K uniform SST warming, indicating a change in the relationship between the 2 variables is likely occurring. 

We can also examine the relationship between $M_{int}$ and $\omega_{int}$ (figure \ref{fig:pe}\textbf{d-f}) and see that the relationship between the 2 does change, most notably in the CAM4-IF. The relationship between $M_{int}$ and $\omega_{int}$  at each gridpoint becomes more steep in the warmer climate, with this implying that the same mass flux in the warmer climate is typically corresponding to a weaker $\omega_{int}$. The opposite is true for the CAM4. However, the most revealing plots are textbf{g-i}, which show the relationship between the $M_{int}$ response and $\omega_{int}$ response at each gridpoint. Generally there is a good relationship between these 2 responses, but with some scatter, but the y-intercepts are the most interesting feature of these graphs. They demonstrate there is an overall offset of the $M_{int}$ compared to the $\omega_{int}$ response. In other words, while the responses of these 2 variables are correlated well spatially, there is an overall increase in mass flux seen, even if the change in $\omega$ is 0 locally. Together, these plots imply that a change in the efficiency of convective mass flux to produce precipitation is occurring in the CAM4-IF models, but no change, or even the opposite, may be occurring in the CAM4 default.
%%%%%%Leave this figure out for now, add later if need be%%%%%
%\begin{figure}[H]
%\centering
%\noindent\includegraphics[width=1\linewidth]{../figures/M_scat-crop.pdf}\hfill
%\caption{Scatterplot of control run $M_{int}$ vs. the fractional response of $M_{int}$ from a 4K SST warming for the 30$^%\circ$S-30$^\circ$N region for each gridpoint. Note the scale is different for each model. Linear regression fit line is in %red.}
%\end{figure}

\begin{figure}[H]
\centering
\noindent\includegraphics[width=1\linewidth]{../figures/pe_scat-crop.pdf}\hfill
\caption{\textbf{a-c}: -$M^{`}$ vs. -$M_{int}$ for each model control (black points) and +4K SST run (red points) for the region 30$^\circ$S-30$^\circ$N, only for grid points where the precipitation rate is $>$ 4 mm/day. \textbf{d-f}: -$\omega_{int}$ vs. -$M_{int}$ for the same grid points as \textbf{a-c} with linear least-squares best fit lines for regions where -$\omega_{int}$ $>$ 0. \textbf{g-i}: $\frac{\delta{M_{int}}}{M_{int}}$ vs. $\frac{\delta{\omega_{int}}}{\omega_{int}}$ for the same grid points as in \textbf{a-f} with the linear least-squares fit line and correlation coefficient indicated. Note the scales vary between plots except for \textbf{g-i}.}
\label{fig:pe}
\end{figure}

In summary the convective mass flux response in the CAM4-IF models does not seem to be consistent with the hydrological cycle argument for mass flux weakening of \citep{held_robust_2006}. While the inferred convective mass flux weakens in the CAM4-IF models, the actual mass flux does not. This could mean that there is a large change in precipitation efficiency in the CAM4-IF model with the 4K SST warming. In the very least, it does seem that the convective scheme does effect the convective mass flux response as this does not occur in the CAM4 with the ZM default scheme. Also, these are +4K SST AMIP experiments, not fully coupled climate change simulations, so it may not be valid to compare directly to the coupled CMIP models. To my knowledge, there is no literature examining the mass flux response in the AMIP4K experiments, so there may be some AMIP4K models that do show an increase in tropical mean convective mass flux.

\subsection{Response of the Walker and Hadley Circulations}
In this section we will primarily focus on the Walker circulation, but we will briefly discuss the Hadley circulation as well.
We see that a weakening of the WC is a very robust response across all the CAM4-IF +4K simulations and the CMIP5 AMIP4K models as well. However, the variation in the magnitude of the response is quite large, ranging from a 36.3\% decrease seen in the MRI-CGCM3 model to only a 3.1\% in the MIROC5 (using $\frac{\delta(\chi_{200})^{*}}{(\chi_{200})}$ as a metric, see table \ref{tab:WC}).

\begin{table}[H]
\caption {Fractional response of various Walker circulation strength metrics for the CAM4/CAM4-IF models and for AMIP4K models which data is available. $\delta({(\omega^{*})_{500}^{\uparrow}})/(\omega^{*})_{500}^{\uparrow}$ is the response of the ascending region, 10S-10N/90E-180E 500 hPa (sigma level 18), upward zonally anomalous omega. $\frac{\delta{P^{*}}}{P^{*}}$ is the repsonse of the zonally anomalous precipitation for the ascending region. $\frac{\delta(Q_{tot})^{*}}{(Q_{tot})^{*}}$ is the response of the zonally anomalous total diabatic heating intergrated thorough the troposphere for the ascending region. Finally, $\frac{\delta(\chi_{200})^{*}}{(\chi_{200})}$ is the response of the 200 hPa zonally anomalous velocity potential for the ascending region. The CAM4-IF models generally weaken the WC more than the CMIP5 AMIP models (based on 200 hPa $\chi^{*}$). The correlation between the $\omega^{*}$ and $\chi^{*}$ measures of WC strength for the CAM4-IF models is 0.86. The correlation between $\frac{\delta(Q_{tot})^{*}}{(Q_{tot})^{*}}$ and $\delta({(\omega^{*})_{500}^{\uparrow}})/(\omega^{*})_{500}^{\uparrow}$ is 0.77, while the correlation between $\frac{\delta(\chi_{200})^{*}}{(\chi_{200})}$ and $\frac{\delta(Q_{tot})^{*}}{(Q_{tot})^{*}}$ is 0.84. The correlation between $\frac{\delta{P^{*}}}{P^{*}}$ and $\delta({(\omega^{*})_{500}^{\uparrow}})/(\omega^{*})_{500}^{\uparrow}$ is 0.66 for the CAM4-IF models and 0.67 for the correlation between $\frac{\delta{P^{*}}}{P^{*}}$ and $\frac{\delta(\chi_{200})^{*}}{(\chi_{200})}$. Finally, the relationship between $\frac{\delta{P^{*}}}{P^{*}}$ and $\frac{\delta(\chi_{200})^{*}}{(\chi_{200})}$ for \textbf{all} models is 0.48. This demonstrates that the total zonally anoamlous diabatic heating response and the zonally anomalous precipitation response are moderately correlated with the response of the WC as measured by either $(\omega^{*})_{500}^{\uparrow}$ or $(\chi_{200})^{*}$.} \label{tab:WC} 
\begin{center}

\begin{tabular}{|p{5.25cm}|p{3cm}|p{1.75cm}|p{2.25cm}|p{2.25cm}|}
\hline
Model&$\delta({(\omega^{*})_{500}^{\uparrow}})/(\omega^{*})_{500}^{\uparrow}$&$\frac{\delta{P^{*}}}{P^{*}}$&$\frac{\delta(Q_{tot})^{*}}{(Q_{tot})^{*}}$&$\frac{\delta(\chi_{200})^{*}}{(\chi_{200})}$\\    \hline
1) Default CAM4&-0.052&0.167&0.083&-0.133\\   \hline
\text{2) CAM4-IF-r}&-0.089&0.135&0.093&-0.148\\ \hline
\text{3) CAM4-IF best T}&-0.155&0.062&0.043&-0.250\\ \hline
4) CAM4-IF 1&-0.266&0.043&-0.015&-0.272\\  \hline
5) CAM4-IF 2&-0.237&0.098&-0.017&-0.296\\  \hline
6) CAM4-IF 3&-0.160&0.071&0.032&-0.225\\  \hline
7) CAM4-IF 4&-0.212&0.125&0.050&-0.258\\  \hline
8) CAM4-IF 5&-0.238&0.093&0.037&-0.290\\  \hline
9) CAM4-IF 6&-0.165&0.112&0.049&-0.222\\  \hline
10) CAM4-IF 7&-0.219&0.105&0.064&-0.194\\  \hline
11) AMIP - bcc-csm1-1 &N/A&0.104&N/A&-0.234\\  \hline
12) AMIP - CanAM4 &N/A&0.108&N/A&-0.074\\  \hline
13) AMIP - CCSM4 &N/A&0.098&N/A&-0.164\\  \hline
14) AMIP - CNRM-CM5 &N/A&0.113&N/A&-0.088\\  \hline
15) AMIP - IPSL-CM5A-LR &N/A&0.018&N/A&-0.065\\  \hline
16) AMIP - IPSL-CM5B-LR &N/A&-0.143&N/A&-0.186\\  \hline
17) AMIP - MIROC5 &N/A&0.298&N/A&-0.031\\  \hline
18) AMIP - MRI-CGCM3 &N/A&-0.144&N/A&-0.363\\  \hline
\end{tabular}

\end{center}
\end{table}


\begin{figure}[H]
\centering
\noindent\includegraphics[width=0.8\linewidth, angle=90]{../figures/200hpachistar_response_ANN.pdf}\hfill
\caption{\textbf{a)-c)}: 200 hPa annual mean $\chi*$ response (shaded) with the control run annual mean climatology in contours. Stipling indicates regions where the response is larger in size than the largest (same-signed) response of the CMIP5 AMIP models.\textbf{d)}: The AMIP4K ensemble mean response of $\chi*$. Note the closer fit of the control and response of the AMIP mean to the CAM4-IF. Stippling indicates regions where the sign of the positive response is the same sign for 10 or more of 12 models and hatching is for regions where the converse is true.}
\end{figure}
-Walker circulation weakens in all 3 models, with the CAM4 having a maximum of weakening over Africa, and the CAM4-IF models seeing a max weakening over the Indian Ocean. 

\begin{figure}[H]
\centering
\noindent\includegraphics[height=0.9\linewidth]{../figures/T_profile_response-crop.pdf}\hfill
\caption{\textbf{a)-d)}: 15$^\circ$S-15$^\circ$N annual mean profiles of temperature response, total diabatic heating response, static stability response and omega response. The total diabatic heating and omega responses are integrated from 1000-100 hPa (pressure-weighted) and the percentage response is indicated in those panels. The static stability response is integrated from 1000-150 hPa as the change above that levels is associated with an increase in tropopause height, not from the general warming and the majority of the upward motion (and hence adiabatic cooling) is below that level. While the fractional static stability response among the 3 models is quite similar, the diabatic heating response has some variation, with a smaller mean increase seen in the CAM4-IF models ($\delta{\sigma}/\sigma - \delta{Q_{tot}}/Q_{tot} \approx \delta{\omega}/\omega$).}
\end{figure}

\begin{figure}[H]
\centering
\noindent\includegraphics[height=1\linewidth]{../figures/Q_response_decomp-crop.pdf}\hfill
\caption{\textbf{a)-d)}: 10$^\circ$S-10$^\circ$N/90$^\circ$E-180$^\circ$E annual mean profiles of zonally anomalous diabatic heating response broken down into contributions from moist processes, shortwave and longwave radiative heating. This represents the components of the response of zonally anomalous energy into the ascending region of the Walker ciruclation. Response are indicated in \textbf{a)}.}

\end{figure}

\begin{figure}[H]
\centering
\noindent\includegraphics[height=1\linewidth,angle=90]{../figures/wp_profiles_response-crop.pdf}\hfill
\caption{\textbf{a}: Thick lines: 15$^\circ$S-15$^\circ$N annual mean profiles of cloud ice and water mixing ratios. Dashed lines: same, but for the response to a 4K SST warming. The tropospheric mean changes in these quantities is quite small for all the models, and most of the change is a shift upward of the ice and water profiles. The largest tropospheric mean changes are seen in the CAM4, with a $\approx$ 12\% increase in cloud water, with a decrease of $\approx$ 11\% in cloud ice. \textbf{b}: Thick lines: control run 15$^\circ$S-15$^\circ$N annual mean profiles of radiative diabatic heating. Dashed lines: same, but for responses. Pressure-weighted vertically integrated percentage responses are indicated (the positive values indicate increased column cooling).}
\end{figure}
-seems that the MASC static stability increases) mechanism is likely the best explanation for weakening WC, with the variation in responses still quite large, likely due to variations in local precipitation and cloud heating responses. 
-CAM4-IF models have a larger radiative cooling response in upper troposphere which is likely related to the larger amount of cloud ice there, particles likely get larger in warmer climate and less absorptive of SW/LW. CAM4 doesn't have much ice to begin with there, so this is not seen in the CAM4.
\subsection{Precipitation efficiency}
Relation \ref{eq:HS} is really only an approximation and neglects those mass fluxes which do not produce preciptiation (presumably from shallow cumulus \citep{held_robust_2006}) and that all the precipitation \textit{produced} eventually reaches the surface and is included in ``$P$". We can define a precipitation efficiency, $\epsilon_{p}$=$P_{prod}/P_{surf}$ where $P_{prod}$ is the precipitation ``production" in the atmosphere and $P_{surf}$ is the rate of precipitation which reaches the surface. Since there is surely at least some evaporation of precipitation before it reaches the surface in the tropics, this efficiency always be less than 1 and likely to be significantly less than 1 in drier regions. Also, $P_{prod}$ is only produced by convective updrafts, so one can then make a modification of \ref{eq:HS} to become: 
\begin{equation}\label{eq:HS1}
\Bigg\langle\frac{\delta{M^{'}_{u}}}{M^{'}_{u}}\Bigg\rangle=\Bigg\langle\frac{\delta{P_{prod}}}{P_{prod}}\Bigg\rangle-\Bigg\langle\frac{\delta{q_{bl}}}{q_{bl}}\Bigg\rangle
\end{equation}
If $\epsilon_{p}$ decreases in a warmer climate, the response in $P_{prod}$ would be larger than $P_{surf}$ and would lead to a smaller implied mass flux decrease, or even an increase, than if one assumed $P_{prod}\equiv{P_{surf}}$ everywhere. 

Unfortunately, an estimate of precipitation production is not possible from data from the CAM4. However, with the IF scheme, it is possible to obtain an estimate of the tropical mean precipitation efficiency using diagnostic output fields from the model by first obtaining an estimate of evaporation of generated condensate in the atmosphere (Ian Folkins, personal communication):
\begin{align}\label{eq:evap}
\text{evap=(up\_wat - up\_wat\_surf)+(an\_snow - an\_snow\_surf)+evap\_det}=P_{prod}-P_{surf}
\end{align}
up\_wat represents the ``starting" amount of rain that is generated by updrafts, up\_wat\_surf the amount of updraft water that reaches the surface. There is also precipitation that falls from convective anvils that starts as snow and eventually a portion reaches the surface as rain in the tropics and this is represented by an\_snow and an\_snow\_surf, respectively. There is also a portion of cloud condensate that does not ``fall" as rain, but is simply detrained and evaporated, represented by evap\_det. With these terms, \ref{eq:evap} allows us to estimate the total loss of condensed water that did not reach the surface as rainfall. This term, plus the surface precipitation rate, should represents $P_{prod}$ that the mass flux creates. For cumulus congestus (shallow cumuli), it is likely that the terms evap\_det and (up\_wat - up\_wat\_surf) are likely large in comparison to the rate of water vapor lifted from the boundary layer as they do not produce much precipitation.

Using \ref{eq:evap}, the tropical mean evaporation rate in the control b.r. version of the CAM4-IF is 1.60$\cdot$10$^{-5}$ kg m$^{-2}$ s$^{-1}$ while it increases to 2.84$\cdot$10$^{-5}$ kg m$^{-2}$ s$^{-1}$ in the +4K run. Adding the respective tropical mean $P_{surf}$ to these 2 numbers gives $P_{prod}$=5.68$^{-2}$ s$^{-1}$, and 7.68$^{-2}$ s$^{-1}$, respectively. This represents a 35.2\% increase, and is much larger than the response of $P_{surf}$, of $\approx$ 19\% (see \ref{tab:title}). Thus, using this 35.2\% increase in $P_{prod}$ gives an implied \textit{increase} in $M^{'}$ of 7.1\%. This is not as high as the observed 16.2\% increase seen in $M_{int}$ (see figure \ref{fig:Mc}), but is a much closer estimate than using $P_{surf}$ to calculate $M^{'}$. This means that $\epsilon_{p}$ decreased from 0.72 in the control run, to 0.63 in the +4K simulation. For the CAM4-IF-t version, tropical mean $P_{prod}$ increases by 34.7\%, very close to the response seen in the b.r. version. This implies a 6.2\% increase in tropical mean $M^{'}$ in this version, which is close to the tropical mean $M_{int}$ calculated in figure \ref{fig:Mc} of 4.1\%. In this version the tropical mean precipitation efficiency decreases from 0.66 to 0.57. Note that this definition of precipitation efficiency is similar to the ``cloud microphysics precipitation efficiency" (CMPE) used in some studies \citep{schoenberg_ferrier_factors_1996,sui_definition_2007}.

\begin{figure}[H]
\centering
\noindent\includegraphics[width=1\linewidth]{../figures/mpq_scat-crop.pdf}\hfill
\caption{textbf{a}: scatter plot of monthly mean tropical mean $\delta{\langle{M_{int}}}\rangle/\langle{M_{int}}\rangle$ and $\delta{\langle{M_{u}^{`}}}\rangle/\langle{M_{u}^{`}}\rangle$ for 300 months of the model runs. Note the slope of the regression line is close to the slope of the one-to-one line (in black) and the points themselves are not far from this line (updraft mass flux data for the CAM4 default was unavailable). Correlation coefficients are indicated in the legend. textbf{b}: same as \textbf{a}, except the x-axis is $\delta{\langle{M^{`}}}\rangle/\langle{M^{`}}\rangle$. Note that in this panel, the correlation coefficients are not as high for the CAM4-IF models as in the previous panel and that the points are much farther from the one-to-one line.   }
\end{figure}

It is possible that horizontal moisture advection by the trade winds is important outside of the deep tropics, and thus moisture may be added or removed horizontally which would complicate the simple model of moisture being transported vertically from the boundary layer then rained out in the same location. Also, since anvils spread away from the updraft region, anvil precipitation may occur at a distance quite removed from where the moisture came from. Again, this would lead to a deviation from the simple model. If there were significant moisture advection out of the 30$^{\circ}$S-30$^{\circ}$N region the calculation of the inferred mass flux could indeed be quite poor, even if the precipitation efficiency were taken into account. If these complicating factors did not change with climate change, it would not matter, but there is no reason to expect that they would not change as we know the tropical circulations and thus the trade winds and upper-level wind patterns would change as well. 

The net increase in tropospheric diabatic heating is not effected by the precipitation efficiency, only by the total precipitation reaching the surface. This is because diabatic cooling is created from this evaporation, and is directly proportional to the mass evaporated, multiplied by $L_{v}$, the latent heat of vaporization (there is surely conceivably some melting and sublimation as well, but the logic for this is the same). Since the converse is also true, $P_{surf}=P_{prod}-E_{trop} \propto \int_{p_{sfc}}^{p_{top}}Q_{cond}dp$ and thus only the change in net precipitation reaching the surface can change the diabatic heating rate due to condensation and thus $\omega$ in the tropics (all else being equal). Therefore, the net precipitation rate response will be more important for the local change in tropical circulation than the change in convective mass flux, if the precipitation efficiency can change.

A possible explanation of a decrease in precipitation efficiency in the CAM4-IF could be a decrease in tropical mean relative humidity in the free troposphere in the tropics. Figure \ref{fig:RH} shows an interesting response of relative humidity in the 3 CAM4 models, with a drying of the upper troposphere and a moistening below 400 hPa. This mainly appears to represent an upward shift of the control run relative humidity profile which has a minimum near 400 hPa. The overall column-integrated tropical mean change in relative humidity is very small ($<$ 1\%) for all 3 models, so it wouldn't appear that the relative humidity is responsible for the precipitation efficiency change seen in the CAM4-IF models. However, the boundary layer ($<$ 850 hPa) relative humidity is decreasing by several \% in the CAM4-IF models. This decrease in boundary layer moisture (sub-LCL/cloud base) is often associated with a decrease in precipitation efficiency \citep{market_precipitation_2003,sherwood_spread_2014}. 
\begin{figure}[H]
\centering
\noindent\includegraphics[width=1\linewidth]{../figures/test_RH-crop.pdf}\hfill
\caption{\textbf{a-c}: 30$^{\circ}$S-30$^{\circ}$N mean relative humidity (\%) response. The magnitude of the tropospheric integrated tropical mean change in relative humidity is $<$ 1\%. }
\label{fig:RH}
\end{figure}

\chapter{Conclusion}
It appears that while the tropical simulations are quite sensitive to the convective scheme used, this sensitivity may be primarily caused by differences in local precipitation and cloud radiative heating, not by differences in the convective mass flux. The convective mass flux can change due to changes in precipitation efficiency and have no effect on the strength of the tropical circulations. Convective mass flux only influences the large-scale upward motion in the sense that it causes condensational heating, but changes in evaporation can offset changes in convective mass flux and the heating that it creates.

Instead of focusing on the convective mass flux per se as mechanism for tropical circulation changes, a better approach would be to examine the changes in the static stability and diabatic heating, as these directly influence the large-scale vertical motions. While some studies have applied this to understanding changes to the WC \citep{sohn_role_2016,ma_mechanisms_2011} and HC \citep{mitas_recent_2006,ma_mechanisms_2011} with GCMs and/or observations and reanalysis data, the effect of changes in cloud radiative heating are not discussed. \citep{he_anthropogenic_2015} and \citep{ma_mechanisms_2011} showed that the uniform increase in SST component is the most important factor for the WC (and perhaps the HC) slowing down due to climate change, but there is significant spread in the weakening of the WC in the AMIP models from this component (from $\approx$ 3-36\%). If the hydrological cycle argument of \citep{held_robust_2006} were the main reason why the WC would weaken, one would not expect much spread beyond in the response to a uniform SST forcing beyond that of the variation in tropical mean precipitation response. Changes in the zonally anomalous diabatic heating due to changes in cloud heating could thus be the culprit (the static stability change is largely constrained by the C-C relation and is distributed quite uniformly in the tropics due to the instability of any temperature gradients). Indeed, \citep{schneider_water_2010} does claim that applying this global mean hydrological cycle argument to the local circulations in the tropics is probably naive.
\\
\section{Future work}
-1-2 paragraphs here to discuss future experiments and improvements to the CAM4-IF.
\\
-would be nice to run coupled simulations - things like MJO/ENSO may interact in unexpected ways with convective scheme.
\\
-is ice cloud in tropics being modeled properly in GCMs? (tropical ice cloud amount seems to be improved in CAM5)
\\
-discuss convection permitting modeling - could be the best way forward as CPU power increases.
\\

%----------------------------------------------------------------------
% END MATERIAL
%----------------------------------------------------------------------

% B I B L I O G R A P H Y
% -----------------------

% The following statement selects the style to use for references.  It controls the sort order of the entries in the bibliography and also the formatting for the in-text labels.
\bibliographystyle{output}
% This specifies the location of the file containing the bibliographic information.  
% It assumes you're using BibTeX (if not, why not?).
\cleardoublepage % This is needed if the book class is used, to place the anchor in the correct page,
                 % because the bibliography will start on its own page.
                 % Use \clearpage instead if the document class uses the ``oneside" argument
\phantomsection  % With hyperref package, enables hyperlinking from the table of contents to bibliography             
% The following statement causes the title "References" to be used for the bibliography section:
\renewcommand*{\bibname}{References}

% Add the References to the Table of Contents
\addcontentsline{toc}{chapter}{\textbf{References}}
%
\bibliography{uw-ethesis}
% Tip 5: You can create multiple .bib files to organize your references. 
% Just list them all in the \bibliogaphy command, separated by commas (no spaces).

% The following statement causes the specified references to be added to the bibliography% even if they were not 
% cited in the text. The asterisk is a wildcard that causes all entries in the bibliographic database to be included (optional).
%\nocite{*}

% The \appendix statement indicates the beginning of the appendices.
\appendix
% Add a title page before the appendices and a line in the Table of Contents
%\chapter*{APPENDICES}
%\addcontentsline{toc}{chapter}{APPENDICES}
%======================================================================
%\chapter[PDF Plots From Matlab]{Appendix stuff}
%\label{AppendixA}
% Tip 4: Example of how to get a shorter chapter title for the Table of Contents 
%======================================================================
%\section{A}

%\section{B}

\end{document}
